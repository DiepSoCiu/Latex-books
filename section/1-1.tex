

\section{Giải phương trình bậc 2}
\subsection{Phương trình bậc 2}
%-----Định nghĩa-----%
\dn{Phương trình bậc 2}{
\textbullet Phương trình bậc 2 có dạng: $ax^2 + bx + c = 0~(a \khac 0)$\mn[-2\baselineskip]{$a \khac 0$ là một điều kiện quan trọng để phương trình trên là phương trình bậc 2 \xd Nếu $a = 0$ \xd $\suyra 0x^2 + bx + c = 0$ \xd $\suyra bx + c = 0$~(Đây là phương trình bậc 1)} $(b,c \thuoc \R)$\mn[4.5\baselineskip]{$b,c \thuoc \R$ nghĩa là 2 số $b, c$ có thể là bất cứ số nào.}.
}
\ex{Phương trình bậc 2}{
    1. $2x^2 = 0$ \xd
    2. $-x^2 + 2x +8 =0$ \xd
    3. $x^2 -9 = 0$
}
\subsection{Phương pháp giải phương trình bậc 2}
\ex{Giải bằng cách sử dụng \textit{hằng đẳng thức}}{
\[ 
\colorbox{themecolor!10!white}{$2x^2 - 5x + 2 = 0$} 
\]
\textbullet\; Bước 1: Chia cả 2 vế cho $a = 2$.  
\[
\frac{2x^2}{2} - \frac{5x}{2} + \frac{2}{2} = \frac{0}{2}
\]
\[
\suyra\; \colorbox{themecolor!10!white}{$x^2 - \frac{5}{2}x + 1 = 0$}
\]
\textbullet\; Bước 2: Biến đổi phương trình trên thành 1 trong 2 hằng đẳng thức là:
\[
\begin{cases}
 A^2 +2AB + B^2 = (A+B)^2 \xd
 A^2 -2AB + B^2 = (A-B)^2
\end{cases}
\]
\[
\suyra \nv{x^2 - 2 \cdot x \cdot \frac{5}{4} + \nt{\frac{5}{4}}^2} - \nt{\frac{5}{4}}^2 +1 = 0
\]
\mn[-5\baselineskip]{$\frac{5}{4} = -\frac{5}{2} : (-2)$} \mn[-3\baselineskip]{Cộng thêm $\nt{\frac{5}{4}}^2$ thì trừ đi $\nt{\frac{5}{4}}^2$ để không làm thay đổi đề bài}
\[
\suyra \nt{x-\frac{5}{4}}^2 - \nt{\frac{5}{4}}^2 +1 = 0
\]
\[
\suyra \nt{x-\frac{5}{4}}^2 - \frac{9}{16} = 0
\]
\[
\suyra \nt{x-\frac{5}{4}}^2 - \nt{\frac{3}{4}}^2 = 0
\]
\mn[-1\baselineskip]{Ta có thể tìm $\nt{\frac{3}{4}}$ bằng cách $\can{\frac{9}{16}}$}
\[
\suyra \nt{x-\frac{5}{4}-\frac{3}{4}}\nt{x-\frac{5}{4}+\frac{3}{4}} = 0
\]
\[
\suyra (x-2)\nt{x-\frac{1}{2}} = 0
\]
\[
\suyra x-2 = 0 ~\hoac~ x-\frac{1}{2} = 0
\]
\[
\suyra x = 2 ~\hoac~ x=\frac{1}{2}
\]
}
\ex{Giải bằng cách sử dụng biệt thức}{
\[ 
\colorbox{themecolor!10!white}{$2x^2 - 5x + 2 = 0$} 
\]
\textbullet Bước 1: Tính biệt thức $\Delta$
\[
\Delta = b^2 -4ac = (-5)^2 - 4.2.2 = 9
\]
\textbullet Bước 2: Nhận xét \xd
Nếu $\Delta < 0 \suyra$ Phương trình vô nghiệm \xd
Nếu $\Delta = 0 \suyra$ Phương trình có nghiệm kép (2 nghiệm giống nhau) $\suyra x = \frac{-b}{2a}$ \xd
Nếu $\Delta > 0 \suyra$ Phương trình có 2 nghiệm phân biệt (2 nghiệm khác nhau)
\[
    x = \frac{-b + \can{\Delta}}{2a} ~ \hoac ~ x = \frac{-b - \can{\Delta}}{2a}
\]
Ta có: \[ \Delta = 9 > 0 \suyra \text{Phương trình có 2 nghiệm phân biệt}\] 
\[
\suyra x = \frac{-(-5) + \can{9}}{2.2} ~ \hoac ~ x = \frac{-(-5) - \can{9}}{2.2}
\]
\[
\suyra x = 2 ~ \hoac ~ x = \frac{1}{2}
\]
}
