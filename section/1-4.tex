
\begin{document}
\section{Vị trí tương đối của 2 đường tròn}
\subsection{Vị trí tương đối là gì ?}
\noindent Để trả lời được câu hỏi này, ta cần trả lời câu hỏi đơn giản sau. Ta đặt tình huống gồm 2 bạn $A$ và $B$ đối thoại với nhau. \xd
\textbf{\textit{Tình huống 1:}} \xd
$A$: Nhà cậu ở đâu vậy ? \xd
$B$: Nhà tớ gần trường Trần Hưng Đạo \xd
\textit{Cách bạn $B$ trả lời vị trí nhà mình chính là vị trí tương đối.} \xdd
\noindent\textbf{\textit{Tình huống 2:}} \xd
$A$: Nhà cậu ở đâu vậy ?\xd
$B$: Nhà tớ ở địa chỉ 76/12/40, đường Thống Nhất, quận Gò Vấp. \xd
\textit{Cách bạn $B$ trả lời vị trí nhà mình chính là vị trí tuyệt đối.} \xdd
\noindent\textbf{\textit{Kết luận:}}\xd
Vị trí tương đối thường mang tính so sánh, chung, tổng quát. \xd
Vị trí tuyệt đối thường mang tính xác định, chính xác.
\subsection{Vị trí tương đối của hai đường tròn}
\vspace{-0.3cm}
\begin{fullpage}
\noindent Bảng dưới đây tóm tắt ví trí tương đối của 2 đường tròn phân biệt $(O_1; R_1)$ và $(O_2; R_2)$\xd
		\begin{tabularx}{\textwidth}{>{\bfseries}+L^L^L^C}
		\arrayrulecolor{themecolor!40!white}
		\toprule\rowstyle{\bfseries}
		Vị trí tương đối        &  \makebox[2cm][c]{Số điểm chung} &          Biểu thức                            &         Hình ảnh                         \\\toprule
		Cắt nhau                & \makebox[2cm][c]{2}              & \makebox[2.1cm][c]{$R_1-R_2<O_1 O_2<R_1+R_2$} & \includegraphics[scale=0.45]{img/cat_nhau.pdf}  \\\midrule
		Tiếp xúc ngoài          &\makebox[2cm][c]{1}               & \makebox[2cm][c]{$O_1 O_2=R_1+R_2$}           & \includegraphics[scale=0.45]{img/tiep_xuc_ngoai.pdf} \\\midrule
		Tiếp xúc trong          & \makebox[2cm][c]{1}              & \makebox[2cm][c]{$O_1 O_2=R_1-R_2$}           & \includegraphics[scale=0.45]{img/tiep_xuc_trong.pdf} \\\midrule
		Ngoài nhau              & \makebox[2cm][c]{0}              & \makebox[2cm][c]{$O_1 O_2>R_1+R_2$}           & \includegraphics[scale=0.45]{img/ngoai_nhau.pdf}  \\\midrule
		Đựng nhau               & \makebox[2cm][c]{0}              & \makebox[2cm][c]{$O_1 O_2<R_1-R_2$}           & \includegraphics[scale=0.45]{img/dung_nhau.pdf} 
	\end{tabularx}
\end{fullpage}



