\newpage
\section{ĐỀ 5}
\begin{smallfont}
\bt[TUYỂN SINH AN GIANG 2024-2025 (1,5)]
{
    Cho hàm số $y=-0,5x^2$ có đồ thị ($P$)
    \begin{enumerate}[label=\alph*)]
        \item Vẽ đồ thị hàm số trên hệ trục $Oxy$.
        \item Tìm các điểm thuộc đồ thị hàm số đã cho có tung độ bằng $-18$.
    \end{enumerate}
}
\xdd 
\bt[TUYỂN SINH TP.HCM 2025-2026 (1)]
{
    Cho phương trình: $2x^2 - 7x + 4 = 0$.
    \begin{enumerate}[label=\alph*)]
        \item Chứng minh phương trình trên có hai nghiệm phân biệt $x_1, x_2$.
        \item Không giải phương trình, hãy tính giá trị biểu thức 
        \[
            A = x_1(3x_2 + x_1) + x_2^2
        \]
    \end{enumerate}
}
\xdd 
\bt[THAM KHẢO (1,5)]
{
    Một hộp có 30 quả bóng được đánh số từ 1 đến 30, các quả bóng có số từ 1 đến 10 được sơn màu cam và các quả bóng còn lại được sơn màu xanh, các quả bóng có kích cỡ và khối lượng như nhau. Lấy ngẫu nhiên một quả bóng trong hộp.
    \begin{enumerate}[label=\alph*)]
        \item Không gian mẫu có bao nhiêu phần tử?
        \item Tính xác suất của mỗi biến cố sau: \xd
              $A:~$"Quả bóng lấy ra có màu cam".\xd 
              $B:~$"Quả bóng lấy ra có màu xanh".\xd 
              $C:~$"Quả bóng lấy ra đánh số tròn chục".\xd 
              $D:~$"Quả bóng lấy ra vừa có màu xanh vừa mang số chia hết cho 3". 
    \end{enumerate}
}
\xdd 
\bt[THAM KHẢO (1)]
{
    Bác Vân có một mảnh đất hình chữ nhật với chiều dài $15~m$ và chiều rộng $12~m$. Bác dự định xây nhà trên mảnh đất đó và dành một phần diện tích đất để làm sân vườn như hình vẽ.   
    \begin{enumerate}[label=\alph*)]
        \item Hãy viết biểu thức thu gọn tính diện tích sân vườn theo $x$.   
        \item Biết diện tích làm nhà là $117~m^2$. Tìm giá trị của $x$. 
    \end{enumerate}~
}
% Hình
\begin{marginfigure}|-3.5cm|
    \centering
    \includegraphics[width=0.8\marginparwidth]{section/DE_THAM_KHAO/DE5/Bai4.pdf}
    \vspace{0.2cm}
    \margincaption{Bài 4}
\end{marginfigure}
\xdd 
\bt[THAM KHẢO (1)]
{
    Hình bên miêu tả một chiếc bình có chứa nước khi được đặt thẳng đứng và khi bị úp ngược, phần chứa nước là phần tô đậm, các số đo được cho như hình vẽ.   
    \begin{enumerate}[label=\alph*)]
        \item Tính thể tích nước trong bình. (kết quả làm tròn đến hàng phần trăm, đơn vị $cm^3$)     
        \item Nếu dùng bình nước này đựng đầy nước rồi rót đầy vào các ly hình lập phương có cạnh $4~cm$ thì rót tối đa đầy được bao nhiêu ly?
    \end{enumerate}
}
% Hình
\begin{marginfigure}|-4.5cm|
    \centering
    \includegraphics[width=\marginparwidth]{section/DE_THAM_KHAO/DE5/Bai5.pdf}
    %\vspace{0.3cm}
    \margincaption{Bài 5}
\end{marginfigure}
\xdd 
\bt[THAM KHẢO (1)]
{
    Bác Thời vay $20~000~000$ đồng từ ngân hàng trong thời hạn một năm với mục đích kinh doanh. Tuy nhiên, do ảnh hưởng của dịch bệnh COVID-19, việc kinh doanh gặp khó khăn, bác không thể thanh toán khoản vay đúng hạn. Vì vậy, bác đã đề nghị ngân hàng gia hạn khoản vay thêm một năm.\xd 
    Ngân hàng đã đồng ý gia hạn khoản vay và gộp toàn bộ số lãi của năm đầu vào vốn để tính lãi cho năm tiếp theo. Sau khi được ngân hàng giải thích rõ cách tính toán, bác Thời đã đồng ý với phương thức này và ký hợp đồng gia hạn khoản vay với ngân hàng. Lãi suất vay vẫn được giữ nguyên như ban đầu.\xd 
    Vì vậy sau hai năm, tổng số tiền bác Thời phải trả $24~200~000$ đồng. Hỏi lãi suất cho vay là bao nhiêu phần trăm một năm? 
}
\xdd 
\bt[TUYỂN SINH HỒ CHÍ MINH 2024-2025 (3)]
{
    Từ điểm $A$ nằm bên ngoài đường tròn ($O; R$), kẻ hai tiếp tuyến $AB, AC$ với đường tròn ($B,C$ là các tiếp điểm), $AO$ cắt $BC$ tại $K$.
    \begin{enumerate}[label=\alph*)]
        \item Chứng minh $ABOC$ là tứ giác nội tiếp và $AO$ là trung trực của đoạn thẳng $BC$.
        \item Gọi $P$ là điểm bất kì thuộc ($O$) sao cho tia $BO$ nằm giữa hai tia $BP$ và $BC$, $H$ là chân đường vuông góc kẻ từ $B$ xuống $PC$, $M$ là trung điểm $BH$ và $PM$ cắt $(O)$ tại $Q$ (khác $P$). Chứng minh $\goc{QMK} = \goc{QCA}$.
        \item Chứng minh $\goc{AQC} = 90\dg$ và $AC = 2R.tan\goc{CPQ}$.
    \end{enumerate}
}
\end{smallfont}
