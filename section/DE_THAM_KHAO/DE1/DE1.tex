\section{ĐỀ 1}
\begin{smallfont}
\bt[THAM KHẢO (1,5)]
{
    Cho Parabol $(P): y = -\frac{1}{2}x^2 $
    \begin{enumerate}[label=\alph*)]
        \item Vẽ đồ thị $(P)$.
        \item Tìm tọa độ giao điểm $M$ thuộc $(P)$ (khác gốc tọa độ) có hoành độ bằng 2 lần tung độ.
    \end{enumerate}
}
\xdd 
\bt[THAM KHẢO (1)]
{
    Cho phương trình $3x^2 + 2x - 5 = 0$
    \begin{enumerate}[label=\alph*)]
        \item Chứng minh phương trình trên có hai nghiệm phân biệt $x_1 , x_2 $.
        \item Không giải phương trình, tính giá trị của biểu thức 
        \[
             T = (x_1 - 2x_2)(x_2 - x_1) + x_2^2 
        \] 
    \end{enumerate}
}
\xdd 
\bt[THAM KHẢO (1,5)]
{
    Thời gian tự học tại nhà của bạn Tú trong một tuần được biểu diễn trong biểu đồ cột sau đây:
    \begin{enumerate}[label=\alph*)]
        \item Tính thời gian trung bình bạn Tú tự học tại nhà mỗi ngày trong một tuần.
        \item Chọn ngẫu nhiên một ngày trong tuần, tính xác suất của các biến cố sau: \xd 
        $A:~$"Ngày được chọn có thời gian tự học tại nhà của bạn Tú lớn hơn 170 phút". \xd 
        $B:~$"Ngày được chọn có thời gian tự học tại nhà của bạn Tú không quá 160 phút".
    \end{enumerate}
    % Hình
    \begin{center}
		\includegraphics[width=0.8\textwidth]{section/DE_THAM_KHAO/DE1/Bieu_do.pdf}
        \captionsetup{hypcap=false}
    \end{center}
}
\xdd 
\bt[THAM KHẢO (1)]
{
    Trên một miếng đất hình chữ nhật có chiều rộng là $x~m$ ($x > 0$), chiều dài là $ 2x +10 ~m $, người ta làm một lối đi hình bình hành có bề rộng là $2~m$. Phần còn lại là hai miếng đất hình thang vuông có diện tích bằng nhau, người ta dự kiến trồng hoa hồng và hoa cúc.
    \begin{enumerate}[label=\alph*)]
        \item Viết biểu thức $A$ biểu diễn diện tích hoa hồng theo $x$.
        \item Người ta dự kiến lát sỏi lối đi, chi phí cho mỗi mét vuông lát sỏi hết 120 nghìn đồng. Hỏi chi phí để làm lối đi là bao nhiêu? Biết diện tích miếng đất trồng hoa hồng là $45~m^2$.
    \end{enumerate}
}
% Hình
\begin{marginfigure}|-4cm|
    \centering
    \includegraphics[width=\marginparwidth]{section/DE_THAM_KHAO/DE1/Bai4.pdf}
    %\vspace{}
    \margincaption{Bài 4}
\end{marginfigure}
\xdd 
\bt[THAM KHẢO (1)]
{
    Trong hình vẽ, 6 lon nước ngọt hình trụ được đặt sát nhau trong thùng carton có dạng hình hộp chữ nhật. Mỗi lon có đường kính $7~cm$ và chiều cao $11~cm$.
    \begin{enumerate}[label=\alph*)]
        \item Tính thể tích thùng carton.
        \item Tính thể tích phần còn trống trong thùng carton khi đựng 6 lon nước ngọt.
    (Các kết quả làm tròn đến hàng đơn vị).
    \end{enumerate}
}
% Hình
\begin{marginfigure}|-4cm|
    \centering
    \includegraphics[width=\marginparwidth]{section/DE_THAM_KHAO/DE1/Bai5.pdf}
    %\vspace{}
    \margincaption{Bài 5}
\end{marginfigure}
\xdd 
\bt[THAM KHẢO (1)]
{
    Để đảm bảo dinh dưỡng trong bữa ăn hằng ngày thì mỗi gia đình 4 thành  viên cần 900 đơn vị protêin và 400 đơn vị Lipit trong thức ăn hằng ngày. Mỗi kilogram thịt bò chứa 800 đơn vị protêin và 200 đơn vị Lipit, còn mỗi kilôgram thịt heo chứa 600 đơn vị protêin và 400 đơn vị Lipit. Biết giá thịt bò là $100~000$ đồng/$kg$ và thịt heo là $70~000$ đồng/$kg$. Tính tổng số tiền mua thịt bò và thịt heo để đảm bảo dinh dưỡng hằng ngày cho 4 người?  
}
\xdd 
\bt[THAM KHẢO (3)]
{
    Từ điểm $A$ nằm ngoài đường tròn ($O;R$) vẽ hai tiếp tuyến $AB, AC$ ($B, C$ là tiếp điểm). Vẽ đường kính $BD$ của đường tròn ($O$). Gọi $K$ là hình chiếu của $C$ trên $BD$, $CK$ cắt $AD$ tại $I$. Gọi $H$ là giao điểm của $OA$ và $BC$.
    \begin{enumerate}[label=\alph*)]
        \item Chứng minh: Tứ giác $ABOC$ nội tiếp và $AO \vuong BC$ .  
        \item Chứng minh: $I$ là trung điểm của $CK$.  
        \item Đường thẳng $BD$ và đường thẳng $AC$ cắt nhau tại $S$ Tia $SI$ cắt $AB$ tại $M$. Giả sử $OA = 2R$. Hãy tính diện tích của tứ giác $AMOC$ theo $R$.  
    \end{enumerate}
}
\end{smallfont}