\section{ĐỀ 4}
\begin{smallfont}
\bt[THAM KHẢO (1,5)]
{
    Cho hàm số $: y = 2x^2 $ có đồ thị $(P)$
    \begin{enumerate}[label=\alph*)]
        \item Vẽ đồ thị $(P)$ trên hệ trục tọa độ.
        \item Tìm những điểm thuộc đồ thị ($P$) có tung độ là 16.
    \end{enumerate}
}
\xdd 
\bt[THAM KHẢO (1)]
{
    Cho phương trình $-x^2 + 10x + 8 = 0$
    \begin{enumerate}[label=\alph*)]
        \item Chứng minh phương trình trên có hai nghiệm phân biệt $x_1 , x_2 $.
        \item Không giải phương trình, tính giá trị của biểu thức 
        \[
             A = (x_1 - x_2)(x_1^2 - x_2^2)
        \] 
    \end{enumerate}
}
\xdd 
\bt[THAM KHẢO (1,5)]
{
    Biểu đồ hình quạt tròn ở hình bên biểu diễn kết quả thống kê (theo tỉ lệ phần trăm) chọn môn thể thao yêu thích nhất trong bốn môn: Cầu lông, Bóng đá, Bóng bàn, Bóng chuyền của 440 học sinh khối 9 ở một trường trung học cơ sở $A$. Mỗi học sinh chỉ được chọn một môn thể thao khi được hỏi ý kiến.  
    \begin{enumerate}[label=\alph*)]
        \item Tính số lượng học sinh cụ thể yêu thích mỗi môn thể thao?
        \item Cô giáo chọn ngẫu nhiên một bạn trong số các bạn học sinh khối 9. Tính xác suất để bạn được chọn không thích môn Cầu lông?  
    \end{enumerate}
}
% Hình
\begin{marginfigure}|-4cm|
    \centering
    \includegraphics[width=\marginparwidth]{section/DE_THAM_KHAO/DE4/Bieu_do.pdf}
    %\vspace{}
    \margincaption{Bài 3}
\end{marginfigure}

\xdd 
\bt[THAM KHẢO (1)]
{
    Mảnh vườn nhà ông Lâm gồm phần đất hình chữ nhật $ABCD$ và phần đất hình thang $CDE$F có các kích thước được cho ở hình     
    \begin{enumerate}[label=\alph*)]
        \item Tính diện tích mảnh vườn nhà ông Lâm theo $x$.  
        \item Sau một vụ canh tác, ông Lâm thu được lợi nhuận 75 triệu đồng. Tìm giá trị của $x$, biết rằng trong vụ đó, mỗi mét vuông đất thu được lợi nhuận $25~000$ đồng.  
    \end{enumerate}~
}
% Hình
\begin{marginfigure}|-3cm|
    \centering
    \includegraphics[width=\marginparwidth]{section/DE_THAM_KHAO/DE4/Bai4.pdf}
    %\vspace{}
    \margincaption{Bài 4}
\end{marginfigure}
\xdd 
\bt[THAM KHẢO (1)]
{
    Một khối hộp chữ nhật (đặc ruột) bằng sắt với kích thước ba cạnh là $12~cm,$ $10~cm,7~cm$ bị khoét bởi một nửa hình trụ có đường kính $4~cm$ và chiều dài $12~cm$.   
    \begin{enumerate}[label=\alph*)]
        \item Tính thể tích của khối kim loại còn lại (làm tròn kết quả đến hàng đơn vị).   
        \item Nếu nấu nóng chảy khối kim loại còn lại này để đúc các viên bi sắt hình cầu có bán kính $1~cm$. Hỏi có thể đúc được nhiều nhất bao nhiêu viên bi sắt như thế?  
    \end{enumerate}
}
% Hình
\begin{marginfigure}|-4.5cm|
    \centering
    \includegraphics[width=0.7\marginparwidth]{section/DE_THAM_KHAO/DE4/Bai5.pdf}
    \vspace{0.3cm}
    \margincaption{Bài 5}
\end{marginfigure}
\xdd 
\bt[THAM KHẢO (1)]
{
    Hai xe ô tô bus đi trên tuyến đường như hình vẽ theo lộ trình từ $ABCDA$ hay $ADCBA$, chạy liên tục không nghỉ và hết lộ trình lại tiếp tục một lộ trình nữa. Cùng một thời điểm, nếu hai ô tô bus cùng xuất phát từ $A$ và đi khác tuyến nhau thì sau 1,2 giờ sẽ  gặp nhau, còn nếu đi cùng tuyến nhau thì ô tô bus thứ hai sẽ vượt ô tô bus thứ nhất sau 6 giờ. Tính thời gian ít nhất mà mỗi xe đi được từ $A$ đến $C$, giả sử rằng vận tốc của các ô tô bus là không đổi trên các đoạn đường của lộ trình.  
}
% Hình
\begin{marginfigure}|-4cm|
    \centering
    \includegraphics[width=\marginparwidth]{section/DE_THAM_KHAO/DE4/Bai6.pdf}
    %\vspace{}
    \margincaption{Bài 6}
\end{marginfigure}
\xdd 
\bt[TUYỂN SINH HỒ CHÍ MINH 2025-2026 (3)]
{
    Từ một điểm $A$ nằm ngoài đường tròn ($O; R$) với $OA = 2R$, kẻ hai tiếp tuyến $AB, AC$ đến đường tròn ($B, C$ là các tiếp điểm). Vẽ đường kính $BD$ của đường tròn ($O$). Gọi $E$ là giao điểm thứ hai của đường thẳng $AD$ với ($O$). Đường thẳng $BC$ và $AO$ cắt nhau tại $H$.
    \begin{enumerate}[label=\alph*)]
        \item Chứng minh rằng tam giác $BED$ vuông và $ABHE$ là tứ giác nội tiếp.
        \item Chứng minh rằng $OD^2 = OH.OA$ và $\goc{HDO} = \goc{HBE}$.
        \item Tính theo $R$ chu vi và diện tích tam giác $DHE$.
    \end{enumerate}
}
\end{smallfont}
