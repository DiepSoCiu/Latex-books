\section{ĐỀ 2}
\begin{smallfont}
\bt[THAM KHẢO (1,5)]
{
    Cho Parabol $(P): y = \frac{1}{2}x^2 $
    \begin{enumerate}[label=\alph*)]
        \item Vẽ đồ thị $(P)$ trên mặt phẳng tọa độ $Oxy$.
        \item Tìm điểm trên $(P)$ có hoành độ là $-3$. 
    \end{enumerate}
}
\xdd 
\bt[THAM KHẢO (1)]
{
    Cho phương trình $x^2 - 5x + 1 = 0$
    \begin{enumerate}[label=\alph*)]
        \item Chứng minh phương trình trên có hai nghiệm phân biệt $x_1 , x_2 $.
        \item Không giải phương trình, tính giá trị của biểu thức 
        \[
             A = \frac{x_1^2 - 5}{x_2} + \frac{x_2^2 - 5}{x_1}
        \] 
    \end{enumerate}
}
\xdd 
\bt[THAM KHẢO (1,5)]
{
    Bác An cân các quả dứa trong gian hàng và ghi lại cân nặng (đơn vị: gam) của từng quả như sau: \xd 
    \begin{tabularx}{\textwidth}{>{}+C^C^C^C^C^C^C^C^C^C}
        \arrayrulecolor{classicgreen!40!white}
        \toprule
         700 & 850 & 852 & 902 & 1000  & 1150  & 1492  & 1159 & 1452 & 1552  \\ \midrule
         905 & 1180 & 850 & 855 & 1020 & 1550  & 1300  & 1470 & 1100 & 1175  \\
        \bottomrule
    \end{tabularx}
    \xd 
    Để thuận tiện cho việc kinh doanh, bác An chia quả dứa thành 3 nhóm theo cân nặng. Gọi $x~gam$ là cân nặng của trái dứa. Hãy hoàn thành bảng số liệu sau:
    \xd 
    \begin{tabularx}{\textwidth}{>{}+L^C^C^C}
        \arrayrulecolor{classicgreen!40!white}
        \toprule\rowstyle{\bfseries}
         Cân nặng & $700 \leq x \leq 1000$  & $1000 \leq x \leq 1300$  & $1300 \leq x \leq 1600$ \\ \midrule
         Số quả dứa & ?  & ?  & ? \\ \midrule
         Tần số tương đối & ? & ? & ? \\ \bottomrule
    \end{tabularx}
}

\xdd 
\bt[THAM KHẢO (1)]
{
    Anh Tâm có một mảnh vườn hình vuông (phần tô đậm) với diện tích bằng $81~m^2$. Anh muốn mở rộng mảnh vườn này thành mảnh vườn hình chữ nhật sao cho các cạnh của hình chữ nhật cách các cạnh của hình vuông tương ứng các khoảng cách là $x;~2x;~3x$ như hình bên, trong đó $x>0$, tính theo đơn vị mét.
    \begin{enumerate}[label=\alph*)]
        \item Viết biểu thức $S$ biểu thị diện tích của mảnh vườn sau khi mở rộng theo $x$.
        \item Biết rằng sau khi mở rộng, diện tích của mảnh vườn tăng thêm $810~m^2$. Tìm giá trị của $x$.
    \end{enumerate}~
}
% Hình
\begin{marginfigure}|-4cm|
    \centering
    \includegraphics[width=0.7\marginparwidth]{section/DE_THAM_KHAO/DE2/Bai4.pdf}
    \vspace{0.4cm}
    \margincaption{Bài 4}
\end{marginfigure}
\xdd 
\bt[THAM KHẢO (1)]
{
    Gạch ống là một sản phẩm được tạo hình thành từ đất sét và nước, được kết hợp lại với nhau theo một công thức chung hợp lý mới có thể tạo ra hỗn hợp dẻo quánh, sau đó chúng  được đổ vào khuôn, rồi đem phơi hoặc sấy khô và cuối cùng là đưa vào lò nung. Một viên gạch hình hộp chữ nhật có kích thước dài $20~cm \times 8~cm \times 8~cm$. Bên trong có bốn lỗ hình trụ bằng nhau có  đường kính đáy là $2,5~cm$.  
    \begin{enumerate}[label=\alph*)]
        \item Tính thể tính đất sét để làm một viên gạch (Lấy $\pi = 3,14$)  
        \item Theo tính toán của kỹ sư xây dựng thì Bác Tư xây một ngôi nhà phải mua 10 thiên gạch (1 thiên = 1000), giá một viên là 2500 đồng. Bác Tư đã mua dư $2\%$ số lượng gạch cần dùng, dự phòng cho hư hao lúc thi công. Tính tổng số tiền Bác Tư mua gạch để xây hoàn thành căn nhà.  
    \end{enumerate}
}
% Hình
\begin{marginfigure}|-5cm|
    \centering
    \includegraphics[width=\marginparwidth]{section/DE_THAM_KHAO/DE2/Bai5.pdf}
    %\vspace{}
    \margincaption{Bài 5}
\end{marginfigure}
\xdd 
\bt[THAM KHẢO (1)]
{
    Buổi sáng, Bình đến một nhà sách mua 10 quyển tập và 5 cây bút hết $146~500$ đồng. Buổi chiều, An cũng đến nhà sách đó mua 10 quyển tập và 10 cây bút thì tổng số tiền phải trả là $168~000$ đồng. Hôm sau, Châu mang $120~000$ đồng mua 8 quyển tập và 2 cây bút. Hỏi Châu có đủ tiền để trả cho cô thu ngân không? Biết giá mỗi quyển tập, mỗi cây bút là bằng nhau và không thay đổi. 
}
\xdd 
\bt[THAM KHẢO (3)]
{
    Cho $\tamgiac ABC$ có ba góc nhọn ($AB < AC$) nội tiếp đường tròn ($O; R$). Hai đường cao $AD, BE$  của $\tamgiac ABC$  cắt nhau tại $H$. Gọi $M , N$  lần lượt là giao điểm của ($O$) với các tia $BE , AD$ ($M$ khác $B$, $N$ khác $A$).  
    \begin{enumerate}[label=\alph*)]
        \item Chứng minh: Tứ giác $ABDE$ nội tiếp và xác định tâm $I$ của đường tròn này, từ đó suy ra $DE // MN$.    
        \item Kẻ đường kính $CK$ của $O$. Chứng minh tứ giác $AKBH$ là hình bình hành và suy ra 3 điểm $H,~I,~K$ thẳng hàng.  
        \item Trong trường hợp $\goc{BCA}= 60\dg$. Chứng minh: $DE = \frac{1}{2}AB$ và tính diện tích hình viên phân giới hạn bởi cung nhỏ $DE$ và dây cung $DE$ của ($I$)  theo $R$.   
    \end{enumerate}
}
\end{smallfont}
