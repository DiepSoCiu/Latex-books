\section{ĐỀ 3}
\begin{smallfont}
\bt[THAM KHẢO (1,5)]
{
    Cho hàm số $: y = \frac{3}{2}x^2 $ có đồ thị $(P)$
    \begin{enumerate}[label=\alph*)]
        \item Vẽ đồ thị $(P)$ trên hệ trục tọa độ.
        \item Tìm những điểm $N$ thuộc đồ thị ($P$) có hoành độ gấp đôi tung độ
    \end{enumerate}
}
\xdd 
\bt[THAM KHẢO (1)]
{
    Cho phương trình $2x^2 - 5x - 7 = 0$
    \begin{enumerate}[label=\alph*)]
        \item Chứng minh phương trình trên có hai nghiệm phân biệt $x_1 , x_2 $.
        \item Không giải phương trình, tính giá trị của biểu thức 
        \[
             A = x_1(x_1^2 + 2022) - x_2(-x_2^3 - 2023) - x_2
        \] 
    \end{enumerate}
}
\xdd 
\bt[THAM KHẢO (1,5)]
{
    Biểu đồ cột kép trong hình biểu diễn số lượng học sinh tham gia giải thi đấu thể thao của một trường trung học cơ sở.
    % Hình
    \begin{center}
		\includegraphics[width=0.8\textwidth]{section/DE_THAM_KHAO/DE3/Bieu_do.pdf}
        \captionsetup{hypcap=false}
    \end{center}
    Chon ngẫu nhiên một học sinh tham gia giải thi đấu thể thao của trường đó. Tính xác suất của mỗi biến cố sau: \xd 
    $A:~$"Học sinh được chọn là nam". \xd 
    $B:~$"Học sinh được chọn thuộc khối 6". \xd 
    $C:~$"Học sinh được chọn là nữ và không thuộc khối 9".
}

\xdd 
\bt[THAM KHẢO (1)]
{
    Một khu đất hình chữ nhật có chiều rộng $35~m$ được chia ra làm hai khu vườn nhỏ để trồng rau. Xung quanh hai khu vườn rau người ta làm lối đi. Lối đi giữa hai vườn rau rộng $1,5~m$ và các lối đi vườn rau còn lại rộng $0,5~m$. Khu vườn rau thứ hai có chiều dài ít hơn khu vườn rau thứ nhất là $15~m$. Gọi $x~m$ là chiều dài của khu vườn rau thứ  nhất.    
    \begin{enumerate}[label=\alph*)]
        \item Viết biểu thức biểu diễn theo $x$ tổng diện tích trồng rau của hai khu vườn.  
        \item Tìm diện tích hai khu trồng rau. Biết rằng diện tích khu đất lớn hơn diện tích trồng rau là $162,5~m^2$.
    \end{enumerate}~
    % Hình
    \begin{center}
		\includegraphics[width=0.7\textwidth]{section/DE_THAM_KHAO/DE3/Bai4.pdf}
        \captionsetup{hypcap=false}
        \captionof{figure}{Bài 4}
    \end{center}
}
\xdd 
\bt[THAM KHẢO (1)]
{
    Gạch chống nóng 6 lỗ còn được gọi là gạch Tuynel, có dạng hình hộp chữ nhật với kích thước $195~mm \times 135~mm \times 90~mm$. Mỗi viên gạch có 6 lỗ rỗng, mỗi lỗ rỗng  này có dạng hình trụ với đường kính đáy $28~mm$.  
    \begin{enumerate}[label=\alph*)]
        \item Lấy $\pi = 3,14$. Hãy tính thể tích nguyên vật liệu để làm nên một viên gạch trên (bỏ qua các rãnh gân của viên gạch).
        \item Một khối đất nung dạng hình hộp chữ nhật với kích thước $2,1~m \times 1,5~m \times 1,5~m$. Người ta dùng khối đất đó để làm gạch. Hỏi cần bao nhiêu khối đất như trên (lấy số  nguyên) để làm ra 10 000 viên gạch biết hao hụt đất nung trong quá trình làm gạch là $10\%$.
    \end{enumerate}
}
% Hình
\begin{marginfigure}|-5cm|
    \centering
    \includegraphics[width=\marginparwidth]{section/DE_THAM_KHAO/DE3/Bai5.pdf}
    %\vspace{}
    \margincaption{Bài 5}
\end{marginfigure}
\xdd 
\bt[THAM KHẢO (1)]
{
    Một hôm hai bố con đang đi dạo trong dãy phố có 6 căn nhà liên tiếp nằm cùng một phía phải của con đường, được đánh số địa chỉ là các số chẵn liên tiếp. Sẵn dịp đang ôn tập kiến thức cho con học ở lớp tiểu học thì ông bố đã đưa ra thử thách “Tìm thương và số dư khi lấy số địa chỉ nhà cuối dãy chia cho số địa chỉ nhà đầu dãy” thì em bé có đáp án thương là 2 và số dư cũng là 2. Biết rằng em bé đã có câu trả lời chính xác. Hãy cho biết số địa chỉ từng căn nhà ở dãy phố đó?
}
\xdd 
\bt[TUYỂN SINH NGHỆ AN 2024-2025 (3)]
{
    Cho tam giác nhọn $ABC$ ($AB < AC$) nội tiếp đường tròn ($O$), hai đường cao $AH$ và $BE$ cắt nhau tại $D$. Gọi $I$ là trung điểm của $BC$.
    \begin{enumerate}[label=\alph*)]
        \item Chứng minh: $ABHE$ là tứ giác nội tiếp.
        \item Kẻ $DK$ vuông góc với $AI (K \in AI)$. Chứng minh: $AK.HI=AH.DK$ và $BC^2 = 4AI.KI$.
        \item Hai tiếp tuyến của đường tròn ($O$) tại $B$ và $C$ cắt nhau tại $M$. Chứng minh: $\goc{BAI} = \goc{CAM}$.
    \end{enumerate}
}
\end{smallfont}
