\newpage
\section{Toán diện tích}
Để giải quyết được các bài toán diện tích ta cần nhớ một số công thức diện tích cơ bản của một số hình phẳng quen thuộc như hình chũ nhật, tam giác, vuông thoi, tròn, \dots
\xd 
\textbf{Hình chữ nhật:}
\[
    \begin{cases}
        S_\textit{hình chữ nhật} = \textit{dài~}.\textit{~rộng} \\[0.5em]
        C_\textit{hình chữ nhật} = 2.(\textit{dài + rộng})
    \end{cases}
\]
\xd
\textbf{Hình vuông:}
\[
    \begin{cases}
        S_\textit{hình vuông} = \textit{cạnh}^2 \\[0.5em]
        C_\textit{hình vuông} = 4.\textit{cạnh}
    \end{cases}
\]
\xd
\textbf{Hình bình hành:}
\[
    \begin{cases}
        S_\textit{hình bình hành} = \textit{đáy~}.\textit{~cao} \\[0.5em]
        C_\textit{hình bình hành} = 2.(\textit{dài + rộng})
    \end{cases}
\]
\xd
\textbf{Hình thang:}
\[
    \begin{cases}
        S_\textit{hình thang} = \frac{1}{2}~ . \textit{~(đáy lớn + đáy bé)~}.\textit{~cao} \\[0.5em]
        C_\textit{hình thang} = \textit{tổng 4 cạnh}
    \end{cases}
\]
\xd
\textbf{Hình thoi:}
\[
    \begin{cases}
        S_\textit{hình thoi} = \frac{1}{2}~ . \textit{~chéo 1~}.\textit{~chéo 2} \\[0.5em]
        C_\textit{hình thoi} = 4.\textit{cạnh}
    \end{cases}
\]
\xd
\textbf{Hình có 2 đường chéo vuông góc:}
\[
    \begin{cases}
        S = \frac{1}{2}~ . \textit{~chéo 1~}.\textit{~chéo 2} \\[0.5em]
        C = \textit{tổng 4 cạnh}
    \end{cases}
\]
\xd
\textbf{Tam giác:}
\[
    \begin{cases}
        S_\tamgiac  = \frac{1}{2}~ . \textit{~đáy~}.\textit{~cao} \\[0.5em]
        C_\tamgiac  = \textit{tổng 3 cạnh}
    \end{cases}
\]
\xd
\textbf{Tam giác vuông:}
\[
    \begin{cases}
        S_{\tamgiac \textit{vuông}}  = \frac{1}{2}~ . \textit{~cgv1~}.\textit{~cgv2} \\[0.5em]
        C_{\tamgiac \textit{vuông}}  = \textit{tổng 3 cạnh}
    \end{cases}
\]
\xd
\textbf{Hình tròn:}
\[
    \begin{cases}
        S_\textit{hình tròn} = \pi . R^2 \\[0.5em]
        C_\textit{hình tròn}  = 2. \pi . R = \pi . \textit{~đường kính} 
    \end{cases}
\]
\xd
\textbf{Hình vành khuyên, hình quạt, hình viên phân:}
\[
    \begin{cases}
        S_\textit{vành khuyên} = \pi . (R^2 - r^2) \\[0.5em]
        S_\textit{hình quạt}  = \frac{\pi . R^2 .n^\circ}{360} \\[0.5em]
        S_\textit{viên phân} = S_\textit{quạt} - S_\tamgiac
    \end{cases}
\]

    \begin{marginfigure}|-12cm|
		\centering
		\includegraphics[width=0.6\marginparwidth]{imgc2/vanhkhuyen.pdf}
		\vspace{0.2cm}
		\margincaption{Hình vành khuyên}
    \end{marginfigure}

    \begin{marginfigure}|-12cm|
		\centering
		\includegraphics[width=0.6\marginparwidth]{imgc2/hinhquat.pdf}
		\vspace{0.2cm}
		\margincaption{Hình quạt}
    \end{marginfigure}

    \begin{marginfigure}|-12cm|
		\centering
		\includegraphics[width=0.6\marginparwidth]{imgc2/vienphan.pdf}
		\vspace{0.2cm}
		\margincaption{Hình viên phân}
    \end{marginfigure}
    
    \begin{smallfont}
        \bt[TUYỂN SINH BÌNH PHƯỚC 2022]
        {
            Cho tam giác $ABC$ vuông tại $A$ có $AC = 12~cm, \goc{B} = 60^\circ$. Hãy tính $\goc{C}, AB, BC$ và diện tích tam giác $ABC$.
        }
        \xdd 
        \bt[TUYỂN SINH BÌNH DƯƠNG 2024-2025]
        {
            Một khu vườn hình chữ nhật có chu vi $200~m$. Do mở rộng đường giao thông nông thôn nên chiều dài khu vườn giảm $8~m$. Tính chiều dài và chiều rộng của khu vườn ban đầu, biết diện tích đất còn lại để trồng cây là $2080~m^2$.
        }
        \xdd 
        \bt[TUYỂN SINH BẾN TRE 2025-2026]
        {
            Một khu vườn hình chữ nhật có chu vi $280~m$. ông An để một lối đi xung quanh vườn rộng $2~m$ (\textit{như hình vẽ bên}). Phần đất còn lại ông An dùng để trồng rau có diện tích $4256~m^2$. Tính chiều dài và chiều rộng của khu vườn đó.
        }
        \begin{marginfigure}|-5cm|
		    \centering
		    \includegraphics[width=0.7\marginparwidth]{imgc2/BT2526.pdf}
            \vspace*{0.2cm}
		    \margincaption{Bài 3}
        \end{marginfigure}

        \xdd 
        \bt[TUYỂN SINH BÌNH THUẬN 2022-2023]
        {
            Ông Bình trang trí một bức tường hình chữ nhật có kích thước $12~m \times 3~m$ bằng cách ốp gạch và vẽ hoa văn. Ông dùng loại gạch dạng viên hình chữ nhật có kích thước $10~cm \times 20~cm$ để ốp. Phần gạch được ốp theo cách: Số viên gạch ở hai hàng kề nhau hơn kém nhau 2 viên, biết rằng hàng dưới cùng có 52 viên, hàng trên cùng có 2 viên và giá thành (gồm cả vật tư và công) cho phần ốp gạch là 400.000 đồng$/m^2$. Giá thành cho phần vẽ hoa văn là 300.000 đồng$/m^2$. Tính số tiền ông Bình phải trả để trang trí bức tường đó. (Biết rằng khoảng trống giữa các viên gạch là không đáng kể).
        }
        \begin{marginfigure}|-7cm|
		    \centering
		    \includegraphics[width=\marginparwidth]{imgc2/BTH2223.pdf}
            \vspace{-0.2cm}
		    \margincaption{Bài 4}
        \end{marginfigure}

        \xdd 
        \bt[TUYỂN SINH BÌNH THUẬN 2023-2024]
        {
            Từ hình vuông đầu tiên, bạn Hùng vẽ hình vuông thứ hai có các đỉnh là trung điểm của các cạnh hình vuông thứ nhất, vẽ tiếp hình vuông thứ ba có các đỉnh là trung điểm của các cạnh hình vuông thứ hai và cứ tiếp tục như vậy (xem hình minh họa bên). Giả sử hình vuông thứ bảy có diện tích bằng $32~cm^2$. Tính diện tích hình vuông thứ năm và cạnh của hình vuông thứ 2.
        }
        \begin{marginfigure}|-7cm|
		    \centering
		    \includegraphics[width=0.6\marginparwidth]{imgc2/BTH2324.pdf}
            \vspace{0.2cm}
		    \margincaption{Bài 5}
        \end{marginfigure}
        \xdd 
        \bt[TUYỂN SINH PHÚ YÊN 2025-2026]
        {
            Để xây dựng công viên từ một mảnh đất hình chữ nhật có chiều dài $30~m$, chiều rộng $20~m$; người ta làm hai lối đi có bề rộng như nhau (hai lối đi này lần lượt song song với chiều dài và chiều rộng của mảnh đất), phần đất còn lại để trồng hoa (hình bên). Xác định bề rộng của lối đi để phần đất trồng hoa có diện tích là $504~m^2$.
        }
        \begin{marginfigure}|-7cm|
		    \centering
		    \includegraphics[width=0.8\marginparwidth]{imgc2/PY2526.pdf}
            \vspace{0.2cm}
		    \margincaption{Bài 6}
        \end{marginfigure}

        \xdd
        \bt[TUYỂN SINH PHÚ YÊN 2023-2024]
        {
            Một khu đất hình chữ nhật có tỷ số hai kích thước là $\frac{2}{3}$. Người ta làm một sân bóng đá mini 5 người ở giữa, chừa lối đi xung quanh (lối đi thuộc khu đất). Lối đi rộng $2~m$ và có diện tích $224~m^2$. Tính các kích thước của khu đất.
        }
        \begin{marginfigure}|-7cm|
		    \centering
		    \includegraphics[width=0.8\marginparwidth]{imgc2/PY2324.pdf}
            \vspace{0.2cm}
		    \margincaption{Bài 7}
        \end{marginfigure}

        \xdd 
        \bt[TUYỂN SINH QUẢNG NGÃI 2019-2020]
        {
            Cho hình vuông $ABCD$. Gọi $S_1$ là diện tích phần giao của hai nửa đường tròn đường kính $AB$ và $AD$. $S_2$ là diện tích phần còn lại của hình vuông nằm ngoài hai nửa đường trong nói trên (như hình vẽ bên). Tính $\frac{S_1}{S_2}$
        }
        \begin{marginfigure}|-5cm|
		    \centering
		    \includegraphics[width=0.6\marginparwidth]{imgc2/QN1920.pdf}
		    \margincaption{Bài 8}
        \end{marginfigure}

        \xdd 
        \bt[TUYỂN SINH CAO BẰNG 2023-2024]
        {
            Một mảnh vườn hình chữ nhật có chu vi là $180~m$. Nếu tăng chiều rộng mảnh vườn lên thêm $20~m$ và giảm chiều dài đi $20~m$ thì diện tích mảnh vườn không thay đổi. Tính chiều dài và chiều rộng ban đầu của mảnh vườn.
        }
        \xdd
        \bt[TUYỂN SINH VĨNH PHÚC 2018-2019]
        {
            Cho một mảnh vườn hình chữ nhật. Biết rằng nếu giảm chiều rộng đi $3~m$ và tăng chiều dài thêm $8~m$ thì diện tích mảnh vườn đó giảm $54~m^2$ so với diện tích ban đầu, nếu tăng chiều rộng thêm $2~m$ và giảm chiều dài đi $4~m$ thì diện tích mảnh vườn đó tăng $32~m^2$ so với diện tích ban đầu. Tính chiều rộng và chiều dài ban đầu của mảnh vườn đó.
        }
        \xdd
        \bt[TUYỂN SINH HỒ CHÍ MINH 2025-2026]
        {
            Một khu vườn hình chữ nhật có chiều dài là $2x~m$ và chiều rộng là $x~m$, $x > 4$. Bác Ba làm một lối đi quanh khu vườn rộng 2 mét như hình vẽ. Phần đất còn lại (phần in đậm) dùng để trồng hoa.
            \begin{enumerate}[label=\alph*)]
		        \item 
		        Viết biểu thức theo $x~m$ biểu diễn diện tích phần đất dùng để trồng hoa và thu gọn biểu thức đó.
		        \item 
		        Giả sử diện tích phần đất trồng hoa là $4800~m^2$. Tính chiều dài và chiều rộng của khu vườn.
	        \end{enumerate}
        }
        \begin{marginfigure}|-12cm|
		    \centering
		    \includegraphics[width=\marginparwidth]{imgc2/HCM2526.pdf}
            \vspace{-0.2cm}
		    \margincaption{Bài 11}
        \end{marginfigure}

        \xdd 
        \bt[TUYỂN SINH HỒ CHÍ MINH 2024-2025]
        {
            Một khu vườn hình chữ nhật có chiều dài là $30~m$ và chiều rộng là $20~m$. Bác Năm làm một lối đi cho khu vườn như hình vẽ (phần tô đậm).
            \begin{enumerate}[label=\alph*)]
		        \item 
		       Hãy viết biểu thức (thu gọn) theo $x$ và $y$ biểu thị diện tích phần còn lại của khu vườn.
		        \item 
		        Tính diện tích phần còn lại của khu vườn khi $x=2,4~m$ và $y=1,8~m$.
	        \end{enumerate}                       
        }
        \begin{marginfigure}|-8cm|
		    \centering
		    \includegraphics[width=\marginparwidth]{imgc2/HCM2425.pdf}
            \vspace{-0.2cm}
		    \margincaption{Bài 12}
        \end{marginfigure}
        \xdd 

        \bt[TUYỂN SINH HƯNG YÊN 2024-2025]
        {
            Một vườn hoa hình tròn có bán kính $OA = 6~~m$. Phía ngoài vườn người ta làm một lối đi xung quanh hình vành khuyên (như hình vẽ). Biết diện tích của lối đi bằng 2 lần diện tích vườn hoa. Khoảng cách giữa 2 điểm $A$, $B$ là (kết quả làm tròn đến chữ số thập phân thứ nhất).
        }
        \begin{marginfigure}|-7cm|
		    \centering
		    \includegraphics[width=0.6\marginparwidth]{imgc2/HY2425.pdf}
            \vspace{0.2cm}
		    \margincaption{Bài 13}
        \end{marginfigure}
        \xdd 

        \bt[TUYỂN SINH QUẢNG NINH 2025-2026]
        {
            Một mảnh đất hình chữ nhật có chiều dài lớn hơn chiều rộng $12~m$. Ở chính giữa mảnh đất người ta làm một vườn hoa hình vuông cạnh bằng $2~m$ (minh họa hình bên). Biết diện tích còn lại của mảnh đất (không tính phần đất làm vườn hoa) là $104~m^2$, tính chiều dài và chiều rộng của mảnh đất.
        }
        \begin{marginfigure}|-5cm|
		    \centering
		    \includegraphics[width=\marginparwidth]{imgc2/QN2526.pdf}
            \vspace{-0.2cm}
		    \margincaption{Bài 14}
        \end{marginfigure}
        \xdd 

        \bt[TUYỂN SINH NINH BÌNH 2025-2026]
        {
            Một mảnh đất hình chữ nhật $ABCD$ có $AB = 30~m$, $BC = 40~m$, có hai vị trí $E$, $F$ cố định lần lượt thuộc cạnh $AB$ và $BC$ sao cho $BE= BF =10~m$. Người ta tạo ra một khu đất hình thang $EFGH (EF // GH)$ để trồng hoa, trong đó các điểm $G, H$ tương ứng thuộc các cạnh $CD$ và $AD$. Hỏi diện tích lớn nhất của khu đất trồng hoa là bao nhiêu mét vuông?
        }
        \begin{marginfigure}|-5cm|
		    \centering
		    \includegraphics[width=0.8\marginparwidth]{imgc2/NB2526.pdf}
            \vspace{0.2cm}
		    \margincaption{Bài 15}
        \end{marginfigure}

        \xdd 
        \bt[THAM KHẢO]
        {
            Ông Thanh thiết kế trong khu vườn hình chữ nhật một khu vực nuôi cá hình tam giác vuông $EAF$ ở một góc khu vườn. Ông căng một sợi dây thẳng từ bờ tường này qua bờ tường kia và đi qua một cái cọc cắm ở vị trí $L$. Khoảng cách từ cọc $L$ đến chiều rộng và chiều dài khu vườn hình chữ nhật lần lượt là $LH = 2~m$, $LG = 3~m$ và biết $GE = 2,5~m$ (theo hình vẽ). Hỏi diện tích khu vực nuôi cá ở góc vườn nhà rộng bao nhiêu mét vuông?
        }
        \begin{marginfigure}|-5cm|
		    \centering
		    \includegraphics[width=0.7\marginparwidth]{imgc2/TK1.pdf}
            \vspace{0.2cm}
		    \margincaption{Bài 16}
        \end{marginfigure}

        \xdd 
        \bt[THAM KHẢO]
        {
            Một viên gạch hình vuông có cạnh là $40~cm$ có hoa văn như hình về bên dưới với $M, N, P. Q$ lần lượt là trung điểm của các cạnh $AB, BC, CD, DA$. Tính diện tích phần tô đậm (làm tròn kết quả đến hàng phần mười).
        }
        \begin{marginfigure}|-6cm|
		    \centering
		    \includegraphics[width=0.7\marginparwidth]{imgc2/TK2.pdf}
            \vspace{0.2cm}
		    \margincaption{Bài 17}
        \end{marginfigure}

        \xdd 
	    \bt[THAM KHẢO]
         {
            Nhà ông Hiền có một mảnh đất hình chữ nhật có chiều rộng là $5~m$ và chiều dài là $50~m$. Do việc mở rộng đường, nên nhà nước đã thu hồi một phần diện tích đất của nhà ông Hiền (phần tam giác, như hình vẽ minh họa).
		\begin{enumerate}[label=\alph*)]
            \item 
            Viết và thu gọn biểu thức $A$ biểu thị theo $x$ diện tích phần đất bị thu hồi của nhà ông Hiền.
            \item 
            Viết và thu gọn biểu thức $B$ biểu thị theo $x$ diện tích phần đất còn lại sau khi bị thu hồi của nhà ông Hiền.
            \item 
            Tìm $A$ và $B$ biết $x=2~m$
        \end{enumerate}	
	    }
        \begin{marginfigure}|-3cm|
		    \centering
		    \includegraphics[width=0.9\marginparwidth]{imgc2/manhdat.pdf}
            \vspace{0.2cm}
		    \margincaption{Bài 18}
        \end{marginfigure}
        \xdd 
        
	    \bt[THAM KHẢO]
        {
        Một cái sân hình chữ nhật có độ dài của một cạnh như hình vẽ. Ở góc sân, người ta làm một cái bồn hoa hình tròn với bán kính $x$ mét $(x>0)$. Biết vòng tròn tiếp xúc với 2 cạnh của hình chữ nhật và khoảng cách từ cạnh (chiều dài) của hình chữ nhật đến đường tròn là $2~m$ (hình minh họa). Cho $\pi = 3,14$
		\begin{enumerate}[label=\alph*)]
            \item 
            Viết biểu thức biểu thị diện tích đất còn lại sau khi đã xây bồn hoa.
            \item 
            Biết bán kính của bồn hoa hình tròn là $3~m$, tính diện tích đất còn lại sau khi xây bồn hoa.
        \end{enumerate}	
	    }
        \begin{marginfigure}|-3cm|
		    \centering
		    \includegraphics[width=0.8\marginparwidth]{imgc2/tron.pdf}
            \vspace{0.2cm}
		    \margincaption{Bài 19}
        \end{marginfigure}
        
        \xdd 
        \bt[THAM KHẢO]
        {
           Một khu trò chơi có dạng hình tròn bán kính $R = 2x~(m)$ với $x > 0$. Ở giữa khu, người ta tạo một sân chơi một hình vuông nội tiếp đường tròn (tức sao cho bốn đỉnh của hình vuông đều nằm trên đường tròn). Phần hình vuông ở giữa là sân chơi, còn phần còn lại giữa đường tròn và hình vuông của khu trò chơi là lối đi lát gạch.
		\begin{enumerate}[label=\alph*)]
            \item 
            Viết biểu thức $S(x)$ biểu diễn diện tích phần sân chơi và thu gọn biểu thức đó.
            \item 
            Biết diện tích phần sân chơi là $800\pi~(m^2)$. Tính diện tích lối đi lát gạch (làm tròn kết quả đến hàng phần trăm).
        \end{enumerate}	
        }
        \begin{marginfigure}|-5cm|
		    \centering
		    \includegraphics[width=0.6\marginparwidth]{imgc2/TK3.pdf}
            \vspace{0.2cm}
		    \margincaption{Bài 20}
        \end{marginfigure}
    \end{smallfont}

