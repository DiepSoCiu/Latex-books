\section{Toán chuyển động có sức cản}
Cái bài toán liên quan đến toán chuyển động có sức cản tập trung xoay quanh vào mối quan hệ giữa 3 đại lượng là $v-$vận tốc, $s-$quãng đường, $t-$thời gian. Ngoài ra còn có các mối quan hệ xoay quanh vận tốc cản. Vì vậy trong chủ đề này ta có một số công thức sau:
\[
    \colorbox{themecolor!10!white}{$v = \frac{s}{t}$}
\]

\[
    \colorbox{themecolor!10!white}{$v_\text{thực tế} = v + v_\text{cản}$ (cùng chiều)}
\]

\[
    \colorbox{themecolor!10!white}{$v_\text{thực tế} = v - v_\text{cản}$ (ngược chiều)}
\]

\newpage
\begin{smallfont}
    \bt[THAM KHẢO]
    {
        Hai ca nô cùng khởi hành từ $A$ đến $B$ cách nhau $85~km$ và đi ngược chiều nhau. Sau 1 giờ 40 phút thì gặp nhau. Tính vận tốc thật của mỗi ca nô, biết rằng vận tốc dòng nước là $3~km/h$ (xem ca nô đi từ $A$ là xuôi dòng) và vận tốc thực tế của ca nô xuôi dòng lớn hơn vận tốc ca nô đi ngược dòng là $9~km/h$.
    }
    \xdd
    \bt[THAM KHẢO]
    {
        Một ca nô chạy xuôi dòng một khúc sông dài $72~km$, rồi chạy ngược dòng khúc sông ấy $64~km$ hết tất cả $7~h$. Nếu ca nô chạy xuôi dòng $120~km$ rồi chạy ngược dòng $32~km$ cũng hết $7~h$. Tính vận tốc riêng của ca nô và vận tốc của nước.
    }
    \xdd
    \bt[TUYỂN SINH HÀ NỘI 2015-2016]
    {
        Một tàu tuần tra chạy ngược dòng $6~km$. Sau đó chạy xuôi dòng $48~km$ trên cùng một dòng sông có vận tốc dòng nước là $2~km/h$. Tính vận tốc của tàu tuần tra khi nước yên lặng, biết thời gian xuôi dòng ít hơn thời gian ngược dòng là 1 giờ.
    }
    \xdd 
    \bt[TUYỂN SINH HÀ NỘI 2000]
    {
        Một ca nô chạy trên sông trong 8 giờ, xuôi dòng $81~km$ và ngược dòng $105~km$. Một lần khác cũng chạy trên khúc sông đó, ca nô này chạy trong 4 giờ, xuôi dòng $54~km$ và ngược dòng $42~km$. Hãy tính vận tốc khi xuôi dòng và ngược dòng của ca nô, biết vận tốc của dòng nước và vận tốc riêng của ca nô không đổi.
    }
    \xdd 
    \bt[THAM KHẢO]
    {
        Một chiếc thuyền xuôi dòng và ngược dòng trên khúc sông dài $40~km$ hết 4 giờ 30 phút. Biết thời gian thuyền xuôi dòng $5~km$ bằng thời gian ngược dòng $4~km$. Tính vận tốc của dòng nước.    
    }
    \xdd 
    \bt[THAM KHẢO]
    {
        Một máy bay chở khách bay thẳng đều từ sân bay $A$ đến sân bay $B$. Khi không có gió và bỏ qua mọi lực cản, máy bay có thể bay ổn định với vận tốc $720~km/h$. Trong một chuyến bay thực tế, trên toàn bộ quãng đường, gió thổi cùng chiều bay với vận tốc $50~km/h$, do không khí loãng ở độ cao lớn và hình dạng thân máy bay, lực cản không khí làm cho vận tốc bay giảm $12\%$ so với vận tốc khi không có gió. Phi công giữ nguyên chế độ bay, không thay đổi công suất động cơ. Hỏi vận tốc của máy bay trong suốt chuyến bay này là bao nhiêu? (Kết quả làm tròn đến hàng phần trăm).
    }
    \xdd 
    \bt[THAM KHẢO]
    {
        Một máy bay bay từ $A$ đến $B$ dài $1200~km$. Gọi vận tốc của máy bay khi không có gió và bỏ qua lực cản là $x~km/h$. Trong lần bay thứ nhất, gió thổi cùng chiều bay, làm tăng vận tốc thêm $80~km/h$, lực cản không khí làm giảm vận tốc $10\%$ vận tốc khi không có gió. Trong lần bay thứ hai (cũng từ $A$ đến $B$), không có gió, do điều kiện thời tiết xấu, lực cản không khí tăng gấp đôi so với lần thứ nhất. Biết rằng thời gian bay lần thứ hai nhiều hơn lần thứ nhất đúng 30 phút.
    }
    \xdd 

\end{smallfont}
