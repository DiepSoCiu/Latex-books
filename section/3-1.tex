\section{Toán chuyển động}
Cái bài toán liên quan đến toán chuyển động tập trung xoay quanh vào mối quan hệ giữa 3 đại lượng là $v-$vận tốc, $s-$quãng đường, $t-$thời gian. Vì vậy trong chủ đề này ta chỉ tập trung xoay quanh một công thức duy nhất:
\[
    \colorbox{themecolor!10!white}{$v = \frac{s}{t}$}
\]

\begin{smallfont}
    \bt[TUYỂN SINH SÓC TRĂNG 2022-2023]
    {
        \textit{Năm 2021 Thủ tướng chính phủ đã phê duyệt dự án xây dựng công trình đường cao tốc Châu Đốc - Cần Thơ - Sóc Trăng, dự án này có ý nghĩa đặc biệt quan trọng, sẽ góp phần phát triển kinh tế xã hội của tỉnh Sóc Trăng nói riêng và khu vực đồng bằng Sông Cửu Long nói chung.} Theo ước tính chiều dài toàn tuyến cao tốc từ Châu Đốc đến Sóc Trăng là $188~km$. Biết rằng vận tốc ô tô đi trên đường cao tốc lớn hơn vận tốc ô tô đi trên quốc lộ là $34~km/h$. Vì vậy nếu ô tô di chuyển trên quãng đường $188~km$ thì việc di chuyển trên đường cao tốc sẽ rút ngắn được 68 phút so với việc di chuyển trên quốc lộ. Tính vận tốc của ô tô khi di chuyển trên đường cao tốc.
    }
    \xdd
    \bt[TUYỂN SINH SÓC TRĂNG 2025-2026]
    {
        Vào lúc 6 giờ sáng, ông An đi ô tô xuất phát từ nhà tại Vị Thanh để đến cơ quan làm việc ở Cần Thơ cách nhà $50~km$. Cùng lúc đó ông Bình đi ô tô từ nhà tại Sóc Trăng đến cùng cơ quan làm việc với ông An cách nhà $60~km.$ Biết rằng ông Bình đi với tốc độ lớn hơn tốc độ của ông An là $10~km/h$ nên đã đến cơ quan cùng lúc với ông An. Hỏi ông An và ông Bình đến cơ quan lúc mấy giờ?
    }
    \xdd
    \bt[TUYỂN SINH BÀ RỊA VŨNG TÀU 2018-2019]
    {
        Hai ô tô khởi hành cùng một lúc từ thành phố $A$ đến thành phố $B$ cách nhau $450~km$ với vận tốc không đổi. Vận tốc xe thứ nhất lớn hơn vận tốc xe thứ hai $10~km/h$ nên xe thứ nhất đến trước xe thứ hai 1,5 giờ. Tính vận tốc mỗi xe.
    }
    \xdd 
    \bt[TUYỂN SINH BÀ RỊA VŨNG TÀU 2022-2023]
    {
        Một người đi xe máy từ điểm $A$ đến địa điểm $B$ trên quãng đường $100~km$. Khi từ $B$ về $A$ người đó đã giảm vận tốc $10~km/h$ so với lúc đi nên thời gian lúc về nhiều hơn thời gian lúc đi là 30 phút. Tính vận tốc của người đó lúc đi.
    }
    \xdd 
    \bt[TUYỂN SINH ĐỒNG THÁP 2018-2019]
    {
        Để chuẩn bị cho mùa giải sắp tới, một vận động viên đua xe ở Đồng Tháp đã luyện tập leo dốc và đổ dốc trên cầu Cao Lãnh. Biết rằng đoạn leo đốc và đổ dốc ở hai bên đầu cầu có độ dài cùng bằng $1~km$. Trong một lần luyện tập, vận động viên khi đổ dốc nhanh hơn vận tốc khi leo dốc là $9~km/h$ và tổng thời gian hoàn thành là 3 phút. Tính vận tốc leo dốc của vận động viên trong lần tập luyện đó.
    }
    \xdd 
    \bt[TUYỂN SINH ĐỒNG THÁP 2025-2026]
    {
        Một người lái xe đi từ $A$ đến $B$ cách nhau $90~km$ với tốc độ và thời gian dự định. Nhưng vì trời mưa, xe đi với tốc độ chậm hơn dự định $15~km/h$ nên thời gian đi đến $B$ nhiều hơn dự định 30 phút. Tính tốc độ dự định và tốc độ thực tế xe đi từ $A$ đến $B$.
    }
    \xdd 
    \bt[TUYỂN SINH ĐỒNG NAI 2021-2022]
    {
        Hằng ngày bạn Mai đi học bằng xe đạp, quãng đường từ nhà đến trường dài $3~km$. Hôm nay, xe đạp hư nên Mai nhờ mẹ chở đi đến trường bằng xe máy với vận tốc lớn hơn vận tốc khi đi xe đạp là $24~km/h$, cùng một thời điểm khởi hành như mọi ngày nhưng Mai đã đến trường sớm hơn 10 phút. Tính vận tốc của bạn Mai khi đi học bằng xe đạp.
    }
    \xdd 
    \bt[TUYỂN SINH KIÊN GIANG 2021-2022]
    {
        Một ô tô chở khách dự định đi từ thành phố Rạch Giá đến thành phố Hà Tiên trên quãng đường dài $90~km$ trong một thời gian đã định. Sau khi đi 1 giờ, ô tô ghé trạm dừng chân nghỉ 10 phút, do đó để đến thành phố Hà Tiên như dự định, người lái xe ô tô tăng vận tốc thêm $9~km/h$. Tính vận tốc lúc đầu của ô tô.
    }
    \xdd 
    \bt[TUYỂN SINH TIỀN GIANG 2019-2020]
    {
        Hai người đi xe đạp từ huyện $A$ đến huyện $B$ trên quãng đường dài $24~km$, khởi hành cùng một lúc. Vận tốc xe của người thứ nhất hơn vận tốc xe của người thứ hai là $3~km/h$ nên người thứ nhất đến huyện $B$ trước người thứ hai là 24 phút. Tính vận tốc của mỗi người.
    }
    \xdd 
    \bt[TUYỂN SINH TIỀN GIANG 2022-2023]
    {
        Một xe tải đi theo hướng từ $A$ đến $B$ cách nhau $210~km$. Sau 2 giờ, cũng trên quãng đường đó, một ô tô khởi hành theo hướng từ $B$ đến $A$ với vận tốc lớn hơn vận tốc xe tải $10~km/h$. Tính vận tốc của xe tải, biết hai xe gặp nhau tại nơi cách $A$ một khoảng bằng $150~km$.
    }
    \xdd 
    \bt[TUYỂN SINH TIỀN GIANG 2020-2021]
    {
        Một người đi xe máy từ địa điểm $A$ đến địa điểm $B$ hết 1 giờ 30 phút, rồi tiếp tục đi từ địa điểm $B$ đến địa điểm $C$ hết 2 giờ. Tìm vận tốc của người đi xe máy trên mỗi quãng đường $AB$ và $BC$, biết quãng đường xe máy đã đi từ $A$ đến $C$ dài $150~km$ và vận tốc của xe máy đi trên quãng đường $AB$ nhỏ hơn vận tốc đi trên quãng đường $BC$ là $5~km/h$
    }
    \xdd 
    \bt[TUYỂN SINH BÌNH DƯƠNG 2018-2019]
    {
        Một người dự định đi xe máy từ tỉnh $A$ đến tỉnh $B$ cách nhau $90~km$ trong một thời gian đã định. Sau khi đi được 1 giờ người đó nghỉ 9 phút. Do đó, để đến tỉnh $B$ đúng hẹn, người ấy phải tăng vận tốc thêm $4~km/h$. Tính vận tốc lúc đầu của người đó.
    }
    \xdd 
    \bt[TUYỂN SINH NGHỆ AN 2019-2020]
    {
        Tình cảm gia đình có sức mạnh thật phi thường. Bạn Vi Quyết Chiến - Cậu bé 13 tuổi quá thương nhớ em trai của mình đã vượt qua một quãng đường dài $180~km$ từ Sơn La đến bệnh viện nhi Trung ương Hà Nội để thăm em. Sau khi đi bằng xe đạp 7 giờ, bạn ấy được lên xe khách và đi tiếp 1 giờ 30 phút nữa thì đến nơi. Biết vận tốc của xe khách lớn hơn vận tốc của xe đạp là $35~km/h$. Tính vận tốc xe đạp của bạn Chiến.      
    }
    \xdd 
    \bt[TUYỂN SINH PHÚ YÊN 2019-2020]
    {
        Một người đi xe máy từ thị trấn Chí Thạnh đến thị trấn Hai Riêng với vận tốc dự định trước. Sau khi đi được $\frac{1}{3}$ quãng đường, vì đoạn đường còn lại xấu nên người đó phải đi với vận tốc nhỏ hơn so với dự định $10~km/h$, do đó đến thị trấn Hai Riêng muộn hơn dự định 18 phút. Tính vận tốc dự định, biết rằng quãng đường từ thị trấn Chí Thạnh đến thị trấn Hai Riêng là $90~km$.
    }
    \xdd 
    \bt[TUYỂN SINH PHÚ YÊN 2022-2023]
    {
        Phú và Yên cùng tham gia cuộc thi ma-ra-tông cự li $10~km$. Trong $4~km$ đầu, cả hai chạy cùng một vận tốc. Trong $6~km$ cuối, Phú tăng vận tốc thêm $2~km/h$. Yên vẫn duy trì vận tốc của mình trong suốt quãng đường đua. Kết quả là Phú về đích sớm hơn Yên 6 phút. Tính vận tốc chạy của Yên.
    }
    \xdd 
    \bt[TUYỂN SINH THỪA THIÊN HUẾ 2024-2025]
    {
        Hai học sinh cùng tham gia một giải chạy với hai cự li khác nhau, cự li của học sinh thứ nhất gấp đôi cự li của học sinh thứ hai (cự li là quãng đường mà người chạy phải hoàn thành). Biết rằng học sinh thứ nhất mất trung bình 5 phút để chạy hết $1~km$, học sinh thứ hai mất trung bình 7 phút đễ chạy hết $1~km$ và thời gian hoàn thành cự li của học sinh thứ nhất nhiều hơn thời gian hoàn thành cự li của học sinh thứ hai là 15 phút. Tính cự li của mỗi học sinh tham gia.
    }
    \xdd 
    \bt[TUYỂN SINH LAI CHÂU 2025-2026]
    {
        Một người đi xe máy từ Than Uyên đến Tam Đường. Khi đi được quãng đường $40~km$ đến Tân Uyên người đó dừng lại nghỉ 20 phút rồi đi tiếp $30~km$ nữa để đến Tam Đường với vận tốc lớn hơn vận tốc khi đi từ Than Uyên đến Tân Uyên là $5~km/h$. Tính vận tốc của người đi xe máy khi đi từ Than Uyên đến Tân Uyên, biết tổng thời gian đi từ Than Uyên đến Tam Đường là 2 giờ.
    }
    \xdd 
    \bt[TUYỂN SINH CAO BẰNG 2024-2025]
    {
        Bạn Hưng đi xe đạp từ nhà đến trường với quãng đường $10~km$. Khi đi từ trường về nhà, vẫn trên cung đường ấy, do lượng xe tham gia giao thông nhiều hơn nên bạn Hưng phải giảm vận tốc $2~km/h$ so với khi đến trường. Vì vậy thời gian về nhà nhiều hơn thời gian đến trường là 10 phút. Tính vận tốc của xe đạp khi bạn Hưng đi từ nhà đến trường và từ trường về nhà.
    }
    \xdd 
    \bt[TUYỂN SINH VĨNH PHÚC 2019-2020]
    {
        Người thứ nhất đi đoạn đường từ địa điểm $A$ đến địa điểm $B$ cách nhau $78~km$. Sau khi người thứ nhất đi được 1 giờ thì người thứ hai đi theo chiều ngược lại vẫn trên đoạn đường đó từ $B$ về $A$. Hai người gặp nhau ở địa điểm $C$ cách $B$ một quãng đường $36~km$. Tính vận tốc của mỗi người, biết rằng vận tốc của người thứ hai lớn hơn vận tốc của người thứ nhất là $4~km/h$ và vận tốc của mỗi người trong suốt đoạn đường là không thay đổi.
    }
    \xdd 
    \bt[TUYỂN SINH LÀO CAI 2022-2023]
    {
        Hai ô tô xuất phát cùng một thời điểm từ địa điểm $A$ đến địa điểm $B$ với vận tốc mỗi ô tô không đổi. Sau 1 giờ quãng đường đi được của ô tô thứ nhất nhiều hơn quãng đường đi được của ôtô thứ hai là $5~km$. Quãng đường đi được của ô tô thứ hai sau 3 giờ nhiều hơn quãng đường đi được của ô tô thứ nhất sau 2 giờ là $35~km$. Tính vận tốc mỗi ô tô.
    }
    \xdd 
    \bt[TUYỂN SINH HỒ CHÍ MINH 2025-2026]
    {
        Từ vị trí $A$ của một công viên có dạng hình vuông $ABCD$ cạnh $a~km$, hai bạn Hòa và Bình bắt đầu chạy bộ cùng lúc với vận tốc không đổi dọc theo các cạnh của hình vuông và theo hai hướng khác nhau. Biết rằng, hai bạn gặp nhau lần thứ nhất tại vị trí $E$ cách $A$ một khoảng bằng $1~km$ và gặp lại nhau lần thứ hai tại vị trí $F$ cách $A$ một khoảng bằng $0,4~km$ như hình vẽ. Gọi $x, y~(km/h)$ lần lượt là vận tốc của Hòa và Bình.
        \begin{enumerate}[label=\alph*)]
		    \item 
		    Chứng minh rằng $\frac{x}{y} = \frac{AB + BC + CE}{AD + DE}$.
		    \item 
		    Tìm giá trị của $a$.
	    \end{enumerate}
    }
    \xdd 
        \begin{marginfigure}|-4.5cm|
		    \centering
		    \includegraphics[width=0.7\marginparwidth]{imgc2/HCM25256.pdf}
		    \vspace{0.2cm}
		    \margincaption{Bài 21}
        \end{marginfigure}
    \bt[TUYỂN SINH BÀ RỊA VŨNG TÀU 2019-2020]
    {
        Có một vụ tai nạn ở vị trí $B$ tại chân của một ngọn núi (chân núi có dạng đường tròn tâm $O$, bán kính $3~km$) và một trạm cứu hộ ở vị trí $A$ (tham khảo hình vẽ). Do chưa biết đường nào để đến vị trí tai nạn nhanh hơn nên đội cứu hộ quyết định điều hai xe cứu thương cùng xuất phát ở trạm đến vị trí tai nạn theo hai cách sau \xd  
        Xe thứ nhất: đi theo đường thẳng từ $A$ đến $B$, do đường xấu nên vận tốc trung bình của xe là $40~km/h$. \xd 
        Xe thứ hai: đi theo đường thẳng từ $A$ đến $C$ với vận tốc trung bình $60~km/h$, rồi đi từ $C$ đến $B$ theo đường cung nhỏ $CB$ ở chân núi với vận tốc trung bình $30~km/h$ (3 điểm $A, O, C$ thẳng hàng và $C$ ở chân núi). Biết đoạn đường $AC$ dài $27~km$ và  $\goc{ABO} = 90^\circ$.
        \begin{enumerate}[label=\alph*)]
		    \item 
		    Tính độ dài quãng đường xe thứ nhất đi từ $A$ đến $B$.
		    \item 
		    Nếu 2 xe cứu thương xuất phát cùng một thời điểm tại $A$ thì xe nào đến vị trí tai nạn trước?
	    \end{enumerate}
    }
    \xdd 
        \begin{marginfigure}|-5.6cm|
		    \centering
		    \includegraphics[width=\marginparwidth]{imgc2/BRVT1920.pdf}
		    \vspace{0.2cm}
		    \margincaption{Bài 22}
        \end{marginfigure}
    \bt[THAM KHẢO]
    {
    Trong một công viên hình tam giác đều $ABC$ cạnh $4~km$, hai vận động viên Lan và Minh xuất phát cùng lúc từ đỉnh $A$. Lan chạy theo chiều kim đồng hồ với vận tốc $6~km/h$ dọc các cạnh $AB$ rồi $BC$, còn Minh chạy ngược chiều kim đồng hồ với vận tốc $8~km/h$ dọc $AC$ rồi $CB$. Biết rằng họ gặp nhau tại điểm $P$ trên cạnh $BC$, cách $B$ một khoảng là $a~km$. Tính $a$ (Kết quả làm tròn đến hàng phần mười).
    }
    \xdd 
    \bt[THAM KHẢO]
    {
    Một thợ lặn đang làm việc tại vị trí $B$ ở độ sâu $30~m$ so với mặt nước (coi mặt nước là vị trí $A$). 
    Do gặp sự cố, thợ lặn buộc phải trồi lên mặt nước theo phương thẳng đứng với vận tốc không đổi. \xd
    Người ta quy ước rằng:
    \begin{itemize}
        \item Khi ở độ sâu $30~m$, mức áp suất tác dụng lên cơ thể thợ lặn là $4$ đơn vị.
        \item Khi lên đến mặt nước, mức áp suất còn $1$ đơn vị.
        \item Trong quá trình trồi lên, áp suất giảm đều theo thời gian.
        \item Nếu áp suất giảm quá $0,1$ đơn vị trong $1$ phút thì được coi là giảm áp đột ngột, gây nguy hiểm.
    \end{itemize}
    Trong thực tế, thợ lặn trồi lên với vận tốc $3~m$/phút.

    \begin{enumerate}[label=\alph*)]
        \item 
        Tính mức giảm áp suất trong $1$ phút của thợ lặn. Từ đó cho biết thợ lặn có rơi vào tình trạng giảm áp đột ngột hay không.
        \item 
        Để đảm bảo an toàn, hãy xác định vận tốc trồi lên lớn nhất (m/phút) mà thợ lặn được phép sử dụng.
    \end{enumerate}
    }
    \xdd
    \bt[THAM KHẢO]
    {
    Quảng trường có hai con đường vuông góc gặp nhau tại điểm $O$ (tạo hình chữ $L$). Điểm $A$ cách $O$ một khoảng $d_1$ theo đường ngang, điểm $B$ cách $O$ một khoảng $d_2$ theo đường dọc.
    
        \begin{itemize}
		    \item 
		    Bạn Lan từ $A$ đi đến $O$ với vận tốc không đổi $v = 4~m/s$.
		    \item 
		    Bạn Minh từ $B$ đi đến $O$ với vận tốc không đổi $v = 10~ m/s$, nhưng khởi hành muộn hơn Lan một khoảng thời gian bằng $\frac{1}{4}$ thời gian Lan đi từ $A$ đến $O$.
	    \end{itemize}

        \begin{enumerate}[label=\alph*)]
		    \item 
		    Để hai bạn gặp nhau đúng tại $O$, điều kiện giữa $d_1$ và $d_2$ phải là gì?
		    \item 
		    Nếu tổng khoảng cách của cả cung đường đó là $50~m$ thì giá trị của $d_1$ và $d_2$ là bao nhiêu mét? (Kết quả làm tròn đến hàng đơn vị).
	    \end{enumerate}
    }
    \xdd 
        \begin{marginfigure}|-6cm|
		    \centering
		    \includegraphics[width=0.7\marginparwidth]{imgc2/quangtruong.pdf}
		    \vspace{0.5cm}
		    \margincaption{Bài 25}
        \end{marginfigure}
    
    \bt[THAM KHẢO]
    {
        Một chiếc máy bay du lịch vô tình bay ngang qua vùng có bão, bão có dạng hình cầu với bán kính là $R = 1000~m$. Biết rằng tại vị trí nguy hiểm (vị trí gần tâm bão nhất) không được nhỏ hơn $200~m$.
        \begin{enumerate}[label=\alph*)]
		    \item 
		    Cho biết vị trí nguy hiểm là ở đâu trên đường bay của máy bay?
		    \item 
		    Tính vận tốc của máy bay khi đang ở vị trí nguy hiểm với khoảng cách tối thiểu. Biết máy bay bay trong vùng bão 5 phút.
	    \end{enumerate}        
    }
        \begin{marginfigure}|-5cm|
		    \centering
		    \includegraphics[width=0.8\marginparwidth]{imgc2/maybay.pdf}
		    \vspace{0.5cm}
		    \margincaption{Bài 26}
        \end{marginfigure}
    \bt[THAM KHẢO]
    {
        Sân trường bạn Thịnh có dạng hình chữ nhật $ABCD$, trong đó $AB = 60~m, BC = 40~m$ và bạn Thịnh muốn đi tới điểm $C$ bằng 2 cách:
        \begin{itemize}
            \item Cách 1: bạn đi theo 2 cạnh là $AB$ và $AC$ với vận tốc là $v~m/s$.
            \item Cách 2: bạn đi thẳng đến $C$ với vận tốc bằng $80\%$ vận tốc đi theo cách 1.
        \end{itemize}
        
        \begin{enumerate}[label=\alph*)]
		    \item Trong 2 cách là cách 1 và cách 2 thì bạn Thịnh sẽ đến $C$ sớm hơn.
		    \item Nếu bạn Thịnh đi một đoạn $x~m$ trên $AB$ rồi cắt vuông góc để đi trên $BC$ sau đó đi thẳng đến $C$ với vận tốc không đổi là $v~m/s$ thì giá trị lớn nhất của $x$ là bao nhiêu để thời gian đi trên cung đường này không lớn hơn thời gian đi bằng cách 2. (Kết quả làm tròn đến hàng phần trăm)
	    \end{enumerate}        
    }
        \begin{marginfigure}|-4.5cm|
		    \centering
		    \includegraphics[width=0.7\marginparwidth]{imgc2/chunhat.pdf}
		    \vspace{0.5cm}
		    \margincaption{Bài 27}
        \end{marginfigure}

\end{smallfont}
