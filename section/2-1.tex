
\section{Tiếp tuyến của đường tròn}
\dn{Tiếp tuyến của đường tròn}{
Tiếp tuyến của đường tròn là đường thẳng chỉ chạm vào đường tròn tại đúng một điểm duy nhất. Điểm đó gọi là \textit{tiếp điểm}.
}
\tc{Tiếp tuyến vuông góc với bán kính của đường tròn tại tiếp điểm.}
\xdd

\section{Các phương pháp dựng hình}
\subsection{Vẽ tiếp tuyến từ một điểm nằm trên đường tròn}
\noindent\textbf{\textit{Bước 1:}} \textit{Dựng bán kính đi qua điểm đó.} \xd
\textbf{\textit{Bước 2:}} \textit{Vẽ một đường thẳng đi qua điểm đó và vuông góc với tiếp tuyến.}
\begin{figure}[h]
	\includegraphics[width=\textwidth]{img/vetieptuyen1.pdf}	
\end{figure}

\subsection{Vẽ tiếp tuyến từ một điểm nằm ngoài đường tròn}
\noindent\textbf{\textit{Bước 1:}} \textit{Vẽ đường thẳng đi qua điểm đó và tiếp xúc với đường tròn một cách tương đối.}\xd
\textbf{\textit{Bước 2:}} \textit{Vẽ bán kính vuông góc với đường thẳng vừa vẽ.}
\begin{figure}[h]
	\includegraphics[width=\textwidth]{img/vetieptuyen2.pdf}	
\end{figure}
\newpage
