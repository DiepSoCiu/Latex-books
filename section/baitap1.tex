\baitap
\vspace{-1cm}
\subsection{Dạng 1: Giải phương trình bậc 2}
\vspace{0.5cm}
\begin{smallfont}
%-----Bài 1-----%
    \bt[CTST Tập 2 Tr.14]{
    Giải các phương trình sau: 
	\begin{multicols}{2}
    \begin{enumerate}[label=\alph*)]
		\item 
		$7x^2 -3x + 2 = 0$
		\item 
		$3x^2 - 2\can{3}x + 1 = 0$
        \item 
        $-2x^2 + 5x +2 = 0$
        \item 
        $5x^2 -12x + 4 = 0$
        \item 
        $5x^2 -2\can{5}x + 1 = 0$
        \item 
        $x^2 -x -20 = 0$
        \item 
        $6x^2 -11x -35 =0$
        \item 
        $16y^2 +24y +9 =0$
	\end{enumerate}
    \end{multicols}
    \textbf{\textit{Gợi ý:}}
    \begin{multicols}{2}
    \begin{enumerate}[label=\alph*)]
		\item 
		$\Delta = -47$
		\item 
        $\Delta = 0$
		\item 
        $\Delta = 41$
        \item 
        $\Delta = 64$
		\item 
        $\Delta = 0$
        \item 
        $\Delta = 81$
        \item 
        $\Delta = 961$
        \item 
        $\Delta = 0$
	\end{enumerate}
    \end{multicols}
    }

\end{smallfont}
\vspace{0.5cm}
\subsection{Dạng 2: Vẽ đồ thị Parabol và phương trình hoành độ giao điểm}
\vspace{0.5cm}
\begin{smallfont}
    %-----Bài 1-----%
    \bt[TUYỂN SINH TP.HCM 2018-2019]{
        Cho Parabol $(P): y=x^2$ và đường thẳng $(d): y=3x-2$.
        \begin{enumerate}[label=\alph*)]
            \item 
            Vẽ $(P)$ và $(d)$ trên cùng hệ trục tọa độ.
            \item 
            Tìm tọa độ giao điểm của $(P)$ và $(d)$ bằng phép tính.
        \end{enumerate}
    }
    \xdd 
    %-----Bài 2-----%
    \bt[TUYỂN SINH TP.HCM 2019-2020]{
        Cho Parabol $(P): y=-\frac{1}{2}x^2$ và đường thẳng $(d): y=x-4$.
        \begin{enumerate}[label=\alph*)]
            \item 
            Vẽ $(P)$ và $(d)$ trên cùng hệ trục tọa độ.
            \item 
            Tìm tọa độ giao điểm của $(P)$ và $(d)$ bằng phép tính.
        \end{enumerate}
    }
    \xdd 
    %-----Bài 3-----%
    \bt[TUYỂN SINH TP.HCM 2020-2021]{
        Cho Parabol $(P): y=\frac{1}{4}x^2$ và đường thẳng $(d): y=-\frac{1}{2}x+2$.
        \begin{enumerate}[label=\alph*)]
            \item 
            Vẽ $(P)$ và $(d)$ trên cùng hệ trục tọa độ.
            \item 
            Tìm tọa độ giao điểm của $(P)$ và $(d)$ bằng phép tính.
        \end{enumerate}
    }
    \xdd 
    %-----Bài 4-----%
    \bt[TUYỂN SINH TP.HCM 2022-2023]{
        Cho Parabol $(P): y=x^2$ và đường thẳng $(d): y=-x+2$.
        \begin{enumerate}[label=\alph*)]
            \item 
            Vẽ $(P)$ và $(d)$ trên cùng hệ trục tọa độ.
            \item 
            Tìm tọa độ giao điểm của $(P)$ và $(d)$ bằng phép tính.
        \end{enumerate}
    }
    \xdd 
    %-----Bài 5-----%
    \bt[TUYỂN SINH TP.HCM 2023-2024]{
        Cho Parabol $(P): y=\frac{x^2}{2}$ và đường thẳng $(d): y=x+4$.
        \begin{enumerate}[label=\alph*)]
            \item 
            Vẽ $(P)$ và $(d)$ trên cùng hệ trục tọa độ.
            \item 
            Tìm tọa độ giao điểm của $(P)$ và $(d)$ bằng phép tính.
        \end{enumerate}
    }
    \xdd 
    %-----Bài 6-----%
    \bt[TUYỂN SINH TP.HCM 2024-2025]{
        Cho Parabol $(P): y=\frac{x^2}{2}$.
        \begin{enumerate}[label=\alph*)]
            \item 
            Vẽ $(P)$ trên hệ trục tọa độ.
            \item 
            Tìm tọa độ các điểm thuộc $(P)$ có tung độ bằng 18.
        \end{enumerate}
        \xdd 
        \textbf{\textit{Gợi ý:}}
        \xdd 
        \textbullet  Tung độ bằng 18 $\suyra y = 18$.
    }
    \xdd 
    %-----Bài 7-----%
    \bt[TUYỂN SINH SÓC TRĂNG 2024-2025]{
        Cho hàm số $y=x^2$ có đồ thị $(P)$
        \begin{enumerate}[label=\alph*)]
            \item 
            Vẽ đồ thị $(P)$ trên mặt phẳng tọa độ $Oxy$.
            \item 
            Trên mặt phẳng tọa độ $Oxy$, xét điểm $A$ có hoành độ bằng 5 và $A$ nằm trên đường thẳng $(d): y=3x+1$. Tìm tọa độ các điểm nằm trên đồ thị $(P)$ có cùng tung độ với điểm $A$.
        \end{enumerate}
        \xdd 
        \textbf{\textit{Gợi ý:}}
        \xdd
        \textbullet  Đề yêu cầu: "Tìm tọa độ các điểm nằm trên đồ thị $(P)$ có cùng tung độ với điểm $A$". \xd 
        $\suyra$ Tìm tung độ điểm $A$. \xd 
        \textbullet Điểm $A$ có hoành độ bằng 5 và $A$ nằm trên đường thẳng $(d): y=3x+1$. \xd 
        $\suyra$ đường thẳng $(d): y=3x+1$ đi qua điểm $A(5; y_A)$. \xd 
        $\suyra$ thay điểm $A$ vào $(d)$ để tìm tung độ điểm $A$ hay $(y_A)$.
        
    }
    \xdd 
    %-----Bài 8-----%
    \bt[TUYỂN SINH TRÀ VINH 2024-2025]{
        Trong mặt phẳng tọa độ $Oxy$, Cho Parabol $(P): y=2x^2$ và đường thẳng $(d):y=-2x+4$.
        \begin{enumerate}[label=\alph*)]
            \item 
            Vẽ đồ thị hai hàm số $(P)$ và $(d)$.
            \item 
            Bằng phép toán tìm tọa độ giao điểm của $(P)$ và $(d)$.
        \end{enumerate}
    }
    \xdd 
    %-----Bài 9-----%
    \bt[TUYỂN SINH TIỀN GIANG 2024-2025]{
        Trong mặt phẳng tọa độ $Oxy$, Cho Parabol $(P): y=2x^2$.
        \begin{enumerate}[label=\alph*)]
            \item 
            Vẽ đồ thị hàm số $(P)$.
            \item 
            Bằng phép toán hãy tìm các điểm thuộc $(P)$ có tung độ bằng 14.
        \end{enumerate}
        \xdd 
        \textbf{\textit{Gợi ý:}}
        \xdd 
        \textbullet  Tung độ bằng 14 $\suyra y = 14$.
    }
    \xdd 
    %-----Bài 10-----%
    \bt[TUYỂN SINH AN GIANG 2024-2025]{
        Cho hàm số $y=-0,5x^2$
        \begin{enumerate}[label=\alph*)]
            \item 
            Vẽ đồ thị hàm số trên hệ trục $Oxy$.
            \item 
            Tìm các điểm thuộc đồ thị hàm số đã cho có tung độ bằng $-18$.
        \end{enumerate}
        \xdd 
        \textbf{\textit{Gợi ý:}}
        \xdd 
        \textbullet  Tung độ bằng $-18$ $\suyra y = -18$.
    }
    \xdd 
    %-----Bài 11-----%
    \bt[TUYỂN SINH NINH THUẬN 2025-2026]{
        Cho hàm số $y=2x^2$ có đồ thị $(P)$.
        \begin{enumerate}[label=\alph*)]
            \item 
            Vẽ đồ thị $(P)$ của hàm số.
            \item 
            Tìm các điểm thuộc Parabol $(P)$ có tung độ bằng 2.
        \end{enumerate}
        \xdd 
        \textbf{\textit{Gợi ý:}}
        \xdd 
        \textbullet  Tung độ bằng 2 $\suyra y = 2$.
    }
    \xdd 
    %-----Bài 12-----%
    \bt[TUYỂN SINH ĐÀ NẴNG 2025-2026]{
        Cho hàm số $y=-\frac{1}{2}x^2$ có đồ thị $(P)$.
        \begin{enumerate}[label=\alph*)]
            \item 
            Vẽ đồ thị $(P)$ của hàm số.
            \item 
            Tìm các điểm thuộc Parabol $(P)$ có tung độ bằng 5 lần hoành độ.
        \end{enumerate}
        \xdd 
        \textbf{\textit{Gợi ý:}}
        \xdd 
        \textbullet  Tung độ bằng 5 lần hoành độ. \xd 
        $\suyra y = 5x$.
    }
    \xdd 
    %-----Bài 13-----%
    \bt{
        Cho hàm số $y=\frac{1}{4}x^2$ có đồ thị $(P)$ và đường thẳng $(d): y=\frac{1}{2}x + 2$.
        \begin{enumerate}[label=\alph*)]
            \item 
            Vẽ $(P)$ và $(d)$ trên cùng một hệ trục tọa độ.
            \item 
            Tìm tọa độ giao điểm của $(P)$ và $(d)$ bằng phép tính.
            \item 
            Tìm phương trình đường thẳng $(d')$ song song với $(d)$ và cắt $(P)$ tại điểm có hoành độ là 2.
        \end{enumerate}
        \xdd 
        \textbf{\textit{Gợi ý:}}
        \xdd 
        \textbullet Phương trình đường thẳng $(d')$ có dạng $y = ax + b$. \xd 
        \textbullet $(d')$ song song với $(d)$ $\suyra a =\frac{1}{2}$ 
        $\suyra y = \frac{1}{2}x + b$. \xd
        \textbullet  $(d')$ cắt $(P)$ tại điểm có hoành độ là 2. \xd 
        $\suyra$ Phương trình hoành độ giao điểm giữa $(P)$ và $(d')$ có nghiệm là $x=2$.
    }
    \xdd 
    %-----Bài 14-----%
    \bt[TUYỂN SINH QUẢNG NAM 2024-2025]{
        Trên mặt phẳng tọa độ $Oxy$, cho đường thẳng $(d): y = ax + b$. Tìm các hệ số $a, b$ biết $(d)$ có hệ số góc bằng $-2$ và $(d)$ cắt Parabol $(P): y = \frac{2}{3}x^2$ tại điểm $M$ có hoành độ dương và tung độ bằng 6.
        \xdd 
        \textbf{\textit{Gợi ý:}}
        \xdd 
        $(d)$ có hệ số góc bằng $-2 \suyra a = -2$ \xd 
        $(d)$ cắt Parabol $(P): y = \frac{2}{3}x^2$ tại điểm $M$ có hoành độ dương và tung độ bằng 6. \xd 
        $\suyra (d)$ cắt Parabol $(P): y = \frac{2}{3}x^2$ tại điểm $M(x_M ; 6)~(x_M > 0)$ \xd 
        $\suyra$ Phương trình hoành độ giao điểm giữa $(P)$ và $(d)$ có nghiệm $x_M > 0$ \xd 
        }
    \xdd 
    %-----Bài 15-----%
    \bt[THI THỬ TRƯỜNG TRẦN QUỐC TOẢN 1 THỦ ĐỨC]{
        \begin{enumerate}[label=\alph*)]
            \item 
            Vẽ đồ thị $(P)$ của hàm số $y=\frac{x^2}{2}$.
            \item 
            Tìm những điểm $A$ thuộc $(P)$ có tung độ gấp đôi hoành độ.
        \end{enumerate}
        \xdd 
        \textbf{\textit{Gợi ý:}}
        \xdd
        \textbullet  Tung độ gấp đôi hoành độ. \xd 
        $\suyra y = 2x$
    }
    \xdd 
    %-----Bài 16-----%
    \bt[THI THỬ TRƯỜNG HỒNG BÀNG TP.HCM 2025-2026]{
    Cho Parabol $(P): y =\frac{x^2}{4}$
    \begin{enumerate}[label=\alph*)]
            \item 
            Vẽ $(P)$ trên hệ trục $Oxy$.
            \item 
            Tìm điểm trên Parabol (P) biết giá trị tuyệt đối của tung độ điểm đó là 9.
        \end{enumerate}
        \xdd 
        \textbf{\textit{Gợi ý:}}
        \xdd
        \textbullet Giá trị tuyệt đối của tung độ điểm đó là 9. \xd 
        $\suyra |y| = 9$ \xd 
        $\suyra$ 
        $
        \bcases{ 
                    y = 9 & $\text{nếu~} y \ge 0$ \cr 
                    y = -9 & $\text{nếu~} y \leq 0$
                }
        $
    }
    \xdd 
    %-----Bài 17-----%
    \bt[TRƯỜNG THỰC HÀNH SÀI GÒN 2025-2026]{
    Cho Parabol $(P): y =-\frac{x^2}{4}$
    \begin{enumerate}[label=\alph*)]
            \item 
            Vẽ $(P)$ trên hệ trục $Oxy$.
            \item 
            Tìm tọa độ các điểm thuộc $(P)$ có hoành độ và tung độ đối nhau.
        \end{enumerate}
        \xdd 
        \textbf{\textit{Gợi ý:}}
        \xdd
        \textbullet  Hoành độ và tung độ đối nhau \xd 
        $\suyra x = -y$
    }
    \xdd 
    %-----Bài 18-----%
    \bt[THAM KHẢO QUẬN 7 TP.HCM 2025-2026]{
    Cho hàm số $y = x^2$ có đồ thị là $(P)$.
    \begin{enumerate}[label=\alph*)]
            \item 
            Vẽ đồ thị $(P)$.
            \item 
            Tìm các điểm $M$ thuộc $(P)$ có tung độ gấp 4 lần hoành độ.
        \end{enumerate}
    }
    \xdd 
    %-----Bài 19-----%
    \bt[TUYỂN SINH KHÁNH HÒA 2023-2024]{
    Trong mặt phẳng tọa độ $Oxy$, cho đường thẳng $(d): y = 6x + 2023$ và Parabol $(P): y = x^2$  
    \begin{enumerate}[label=\alph*)]
            \item 
            Vẽ đồ thị Parabol $(P)$.
            \item 
            Chứng minh $(d)$ cắt $(P)$ tại hai điểm phân biệt.
        \end{enumerate}
        \xdd 
        \textbf{\textit{Gợi ý:}}
        \xdd
        \textbullet Chứng minh $(d)$ cắt $(P)$ tại hai điểm phân biệt. \xd 
        $\suyra$ Phương trình hoành độ giao điểm của $(d)$ và $(P)$ có 2 nghiệm phân biệt.
    }
    \xdd 
    %-----Bài 20-----%
    \bt[TUYỂN SINH TIỀN GIANG 2022-2023]{
    Trong mặt phẳng tọa độ $Oxy$, cho đường thẳng $(d): y = -2x + 3$ và Parabol $(P): y = x^2$  
    \begin{enumerate}[label=\alph*)]
            \item 
            Vẽ Parabol $(P)$. Bằng phép tính, tìm tọa độ giao điểm của $(P)$ và $(d)$.
            \item 
            Viết phương trình đường thẳng $(d')$ song song với $(d)$ và tiếp xúc với $(P)$. Tính tọa độ tiếp điểm $M$ của $(d')$ và $(P)$. 
        \end{enumerate}
        \xdd 
        \textbf{\textit{Gợi ý:}}
        \xdd
        \textbullet Phương trình đường thẳng $(d')$ có dạng là $y = ax + b$ \xd 
        \textbullet Đường thẳng $(d')$ song song với $(d)$ \xd 
        $\suyra a = -2$ \xd 
        $\suyra (d'): y = -2x + b $ \xd 
        \textbullet Đường thẳng $(d')$ tiếp xúc với $(P)$ \xd 
        $\suyra$ Phương trình hoành độ giao điểm của $(d')$ và $(P)$ có nghiệm kép.
    }
\end{smallfont}
\vspace{0.5cm}
\subsection{Dạng 3: Định lý Viète không chứa tham số}
\vspace{0.5cm}
\begin{smallfont}
    %-----Bài 1-----%
    \bt[TUYỂN SINH TP.HCM 2018-2019]{
        Cho phương trình: $3x^2 -x -1 = 0$ có 2 nghiệm là $x_1; x_2$. \xd 
        Không giải phương trình, hãy tính giá trị biểu thức $A = x_1^2 + x_2^2$.
        \xdd 
        \textbf{\textit{Gợi ý:}}
        \xdd
        \textit{Cách 1:} \xd 
        $A = x_1^2 + x_2^2 = (x_1 + x_2)^2 -2x_1x_2$ \xd 
        \textit{Cách 2:} \xd
        Ta có: $x_1; x_2$ là 2 nghiệm của phương trình $3x^2 -x -1 = 0 $ \xd 
        $\suyra$
        $
        \bcases{ 
                    3x_1^2 -x_1 -1 = 0 \cr 
                    3x_2^2 -x_2 -1 = 0
                }
        $ $\suyra$
        $\bcases{ 
                    3x_1^2 = x_1 + 1 \cr 
                    3x_2^2 = x_2 + 1 
                }
        $ $\suyra$
        $\bcases{ 
                    x_1^2 =\frac{x_1 + 1}{3}  \cr 
                    x_2^2 = \frac{x_2 + 1}{3} 
                }
        $ \xd 
        $\suyra A = x_1^2 + x_2^2 = \frac{x_1 + 1}{3} + \frac{x_2 + 1}{3} = \frac{x_1 + x_2 + 1 + 1}{3} = \frac{x_1 + x_2 + 2}{3}$ 

    }
    \xdd
    %-----Bài 2-----%
    \bt[TUYỂN SINH TP.HCM 2019-2020]{
        Cho phương trình: $2x^2 -3x -1 = 0$ có 2 nghiệm là $x_1; x_2$. \xd 
        Không giải phương trình, hãy tính giá trị biểu thức $A = \frac{x_1-1}{x_2+ 1} + \frac{x_2 - 1}{x_1 + 1}$.
        \xdd 
        \textbf{\textit{Gợi ý:}}
        \xdd
        $A = \frac{x_1-1}{x_2+ 1} + \frac{x_2 - 1}{x_1 + 1} = \frac{(x_1 - 1)(x_1 + 1) + (x_2 -1)(x_2 + 1)}{(x_2 + 1)(x_1 +1)}$
    }
    \xdd
    %-----Bài 3-----%
    \bt[TUYỂN SINH TP.HCM 2020-2021]{
        Cho phương trình: $2x^2 -5x -3 = 0$ có 2 nghiệm là $x_1; x_2$. \xd 
        Không giải phương trình, hãy tính giá trị biểu thức $A = (x_1 + 2x_2)(x_2 + 2x_1)$.
    }
    \xdd
    %-----Bài 4-----%
    \bt[TUYỂN SINH TP.HCM 2022-2023]{
        Cho phương trình: $2x^2 -4x -3 = 0$ có 2 nghiệm là $x_1; x_2$. \xd 
        Không giải phương trình, hãy tính giá trị biểu thức $A = (x_1 - x_2)^2$.
    }
    \xdd
    %-----Bài 5-----%
    \bt[TUYỂN SINH TP.HCM 2023-2024]{
        Cho phương trình: $2x^2 -13x -6 = 0$ có 2 nghiệm là $x_1; x_2$. \xd 
        Không giải phương trình, hãy tính giá trị biểu thức $A = (x_1 + x_2)(x_1 + 2x_2) - x_2^2$.
    }
    \xdd
    %-----Bài 6-----%
    \bt[TUYỂN SINH TP.HCM 2024-2025]{
        Cho phương trình: $3x^2 -4x -2 = 0$ có 2 nghiệm là $x_1; x_2$. \xd 
        Không giải phương trình, hãy tính giá trị biểu thức $A = x_1x_2^2 + x_2(x_1^2 + 2) + 2x_1$.
        \xdd 
        \textbf{\textit{Gợi ý:}}
        \xdd
        $A = x_1x_2^2 + x_2(x_1^2 + 2) + 2x_1 = x_1x_2^2 + x_2x_1^2 + 2x_2 + 2x_1 \xd = x_1x_2(x_1 + x_2) + 2(x_1+x_2)$
    }
    \xdd
    %-----Bài 7-----%
    \bt[TUYỂN SINH TP.HCM 2025-2026]{
        Cho phương trình: $2x^2 - 7x + 4 = 0$.
        \begin{enumerate}[label=\alph*)]
            \item 
            Chứng minh phương trình trên có hai nghiệm phân biệt $x_1, x_2$.
            \item
            Không giải phương trình, hãy tính giá trị biểu thức $A = x_1(3x_2 + x_1) + x_2^2$.
        \end{enumerate}
    }
    \xdd
    %-----Bài 8-----%
    \bt[TUYỂN SINH TRÀ VINH 2025-2026]{
        Cho phương trình: $2x^2 + 4x - 1 = 0$.
        \begin{enumerate}[label=\alph*)]
            \item 
            Chứng minh phương trình trên có hai nghiệm phân biệt.
            \item
            Không giải phương trình, hãy tính giá trị biểu thức $A = \frac{x_2}{x_1} - \frac{2}{x_2}$.
        \end{enumerate}
        \xdd 
        \textbf{\textit{Gợi ý:}}
        \xdd
        $A = \frac{x_2}{x_1} - \frac{2}{x_2} = \frac{x_2^2 - 2x_1}{x_1x_2}$ \xd 
        \textit{Cách 1:} \xd 
        $x_2$ là một nghiệm của phương trình $2x^2 + 4x - 1 = 0$ nên ta có: \xd  $2x_2^2 +4x_2 - 1 = 0$ \xd 
        $\suyra 2x_2^2 = -4x_2 + 1 \xd \suyra x_2^2 = \frac{-4}{2}x_2 + \frac{1}{2}  = -2x_2  + \frac{1}{2}$ \xd
        $\suyra A = \frac{-2x_2  + \frac{1}{2} - 2x_1}{x_1x_2}$ \xd 
        $\suyra A = \frac{-2x_2  - 2x_1 + \frac{1}{2} }{x_1x_2}$ \xd
        $\suyra A = \frac{-2(x_2  + x_1) + \frac{1}{2} }{x_1x_2}$ \xd
        \textit{Cách 2:}
        \xdd
        $A = \frac{x_2}{x_1} - \frac{2}{x_2} = \frac{x_2^2 - 2x_1}{x_1x_2}$ \xd
        Từ định lí Viète, ta có: \xd 
        $
        \begin{cases}
            x_1 + x_2 = -2 \xd 
            x_1x_2 = \frac{-1}{2} 
        \end{cases}
        $ \xd 
        Nhận thấy có số $-2$ trong biểu thức $A$ nên ta thay $x_1 + x_2 = -2$ \xd 
        $\suyra A = \frac{x_2^2 - 2x_1}{x_1x_2} = \frac{x_2^2 + (x_1 + x_2 ).x_1}{x_1x_2} $ \xd
        $\suyra A = \frac{x_2^2 + x_1^2 + x_1x_2}{x_1x_2}$
    }
    \xdd
    %-----Bài 9-----%
    \bt[TUYỂN SINH ĐỒNG THÁP 2025-2026]{
        Gọi $x_1, x_2$ là hai nghiệm của phương trình $x^2 - x - 12 = 0$. Không giải phương trình, hãy tính giá trị của biểu thức: $A = x_1 + x_2 - 2x_1x_2$.
    }
    \xdd
    %-----Bài 10-----%
    \bt[TUYỂN SINH ĐỒNG NAI 2023-2024]{
        Cho phương trình $3x^2 + 5x - 1 = 0$ có hai nghiệm $x_1, x_2$. Hãy tính giá trị của biểu thức: $T = 6x_1 -7x_1x_2 +6x_2$.
    }
    \xdd
    %-----Bài 11-----%
    \bt[TUYỂN SINH ĐỒNG NAI 2021-2022]{
        Cho phương trình $x^2 + 5x - 4 = 0$. Gọi $x_1; x_2$ là hai nghiệm của phương trình. Không giải phương trình, hãy tính giá trị của biểu thức: $Q = x_1^2 + x_2^2 +6x_1x_2$.
    }
    \xdd
    %-----Bài 12-----%
    \bt[TUYỂN SINH CẦN THƠ 2025-2026]{
        Gọi $x_1, x_2$ là hai nghiệm của phương trình $2x^2 - 6x + 1 = 0$. Tính giá trị của biểu thức: $M = \frac{1}{x_1} + \frac{1}{x_2} + \frac{2}{x_1x_2}$.
    }
    \xdd
    %-----Bài 13-----%
    \bt[TUYỂN SINH KIÊN GIANG 2020-2021]{
        Gọi $x_1, x_2$ là hai nghiệm của phương trình $5x^2 + 12x - 30 = 0$. Không giải phương trình hãy tính giá trị biểu thức $A = 4x_1x_2 - x_1^2 - x_2^2.$
    }
    \xdd
    %-----Bài 14-----%
    \bt[TUYỂN SINH BÌNH PHƯỚC 2025-2026]{
        Gọi $x_1, x_2$ là hai nghiệm của phương trình $x^2 - 3x + 2 = 0$. Không giải phương trình hãy tính giá trị biểu thức $P = x_1^3 + 3x_2^2 + 2x_1 + 2011.$
    }
    \xdd
    %-----Bài 15-----%
    \bt[TUYỂN SINH TIỀN GIANG 2025-2026]{
        Gọi $x_1, x_2$ là hai nghiệm của phương trình $x^2 + 17x - 6 = 0$. Không giải phương trình hãy tính giá trị biểu thức $T = (x_1 + 1)(x_2 + 1)$.
    }
    \xdd
    %-----Bài 16-----%
    \bt[TUYỂN SINH TIỀN GIANG 2024-2025]{
        Cho phương trình $x^2 + 8x - 5 = 0$ có 2 nghiệm phân biệt $x_1, x_2$. Không giải phương trình, hãy tính giá trị biểu thức $B = x_1^2 x_2 + x_1 x_2^2 - 3x_1 x_2$.
    }
    \xdd
    %-----Bài 17-----%
    \bt[TUYỂN SINH BẾN TRE 2024-2025]{
        Gọi $x_1, x_2$ là hai nghiệm của phương trình $x^2 - 3x - 10 = 0$. Không giải phương trình hãy tính giá trị biểu thức $A = \frac{x_1 + 1}{x_2} + \frac{x_2 + 1}{x_1}$.
    }
   \xdd
    %-----Bài 18-----%
    \bt[TUYỂN SINH NGHỆ AN 2022-2023]{
        Cho phương trình $x^2 +3x -1 = 0$ có hai nghiệm $x_1, x_2$. Không giải phương trình, hãy tính giá trị của biểu thức $T = \frac{3|x_1 - x_2|}{x_1^2x_2 + x_1x_2^2}$.
        \xdd 
        \textbf{\textit{Gợi ý:}}
        \xdd
        $T = \frac{3|x_1 - x_2|}{x_1^2x_2 + x_1x_2^2} = \frac{3|x_1 - x_2|}{x_1x_2(x_1 + x_2)}$ \xd
        Viète: \xd
        $T = \frac{3|x_1 - x_2|}{-1(-3)} = |x_1 - x_2|$ \xd
        $\suyra T^2 = (x_1 - x_2)^2$
    }
   \xdd
    %-----Bài 19-----%
    \bt[TUYỂN SINH HÀ TĨNH 2025-2026]{
        Cho phương trình $x^2 +3x -1 = 0$ có hai nghiệm $x_1, x_2$. Không giải phương trình, hãy tính giá trị của biểu thức $T = (x_1 + 3)^2 + (x_2 + 3)^2$.
    }
   \xdd
    %-----Bài 20-----%
    \bt[TUYỂN SINH HUẾ 2025-2026]{
        Cho phương trình $x^2 -3x +1 = 0$.
        \begin{enumerate}[label=\alph*)]
            \item 
            Chứng minh phương trình trên có hai nghiệm phân biệt $x_1, x_2$.
            \item
            Không giải phương trình, hãy tính giá trị biểu thức $P = \frac{2}{x_2 -1} + \frac{x_2}{x_1 - 1}$.
        \end{enumerate}
        \xdd 
        \textbf{\textit{Gợi ý:}}
        \xdd
        $P = \frac{2}{x_2 -1} + \frac{x_2}{x_1 - 1} = \frac{2.(x_1 - 1) + x_2.(x_2 - 1)}{(x_2 - 1 )(x_1 - 1)} = \frac{2x_1 - 2 + x_2^2 - x_2}{(x_2 - 1 )(x_1 - 1)}$ \xd
        $x_2$ là một nghiệm của phương trình $x^2 - 3x + 1 = 0$ nên ta có: $x_2^2 - 3x_2 + 1 = 0$ \xd 
        $\suyra x_2^2 = 3x_2 -1$ \xd 
        $\suyra P = \frac{2x_1 - 2 + x_2^2 - x_2}{(x_2 - 1 )(x_1 - 1)} = \frac{2x_1 - 2 + 3x_2 -1 - x_2}{(x_2 - 1 )(x_1 - 1)}$
    }              
\end{smallfont}
